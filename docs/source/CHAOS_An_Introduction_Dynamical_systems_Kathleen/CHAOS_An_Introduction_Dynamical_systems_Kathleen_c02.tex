\documentclass[12pt]{article}
\begin{document}

\textbf{Two-Dimensional Maps }

\hfill

IN CHAPTER 1 we developed the fundamental properties of one-dimensional dynamics. The concepts of periodic orbits, stability, and 
sensitive dependence of orbits are most easily understood in that context.

In this chapter we will begin the transition from one-dimensional to many dimensional dynamics. The discussion centers around 
two-dimensional maps, since much of the new phenomena present in higher dimensions appears there in its simplest form. For example, 
we will expand our classification of one-dimensional fixed points as sinks and sources to include saddles, which are contracting in 
some directions and expanding in others.


2.1 M ATHEMATICAL M ODELS

Maps are important because they encode the behavior of deterministic systems. We have discussed population models, which have a 
single input (the present population) and a single output (next year’s population). The assumption of determinism is that the output 
can be uniquely determined from the input. Scientists use the term “state” for the information being modeled. The state of the 
population model is given by a single number—the population—and the state space is one-dimensional. We found in Chapter 1 that even a 
model where the state runs along a one-dimensional axis, as G(x)  4x(1  x), can have extremely complicated dynamics.

The changes in temperature of a warm object in a cool room can be modeled as a one-dimensional map. If the initial temperature 
difference between the object and the room is D(0), the temperature difference D(t) after t minutes is

D(t)  D(0)e Kt ,  

(2.1)

where K>0 depends on the specific heat of the object. We can call the model

f(x)  e K x

(2.2)

the “time-1 map”, because application of this map advances the system one minute. Since e K is a constant, f is a linear 
one-dimensional map, the easiest type we studied in Chapter 1. It is clear that the fixed point x  0 is a sink because |e K |  1. 
More generally, we could consider the map that advances the system T time units, the time-T map. For any fixed time T, the time-T map 
for this example is the linear one-dimensional map f(x)  e KT x, also written as

x ↦ −→ e KT x.

(2.3)

Formula (2.1) is derived from Newton’s law of cooling, which is the differential equation

dD  kD, dt

(2.4)

with initial value D(0). There is a basic assumption that the body is a good conductor so that at any instant, the temperature 
throughout the body is uniform. Equation (2.3) is an example of the derivation of a map as the time-T map of a differential equation.

In the examples above, the state space is one-dimensional, meaning that the information needed to advance the model in time is a 
single number. This is in effect the definition of state in mathematical modeling: the amount of information
needed to advance the model. Nothing else is relevant for the purposes of the model, so the state is completely sufficient to 
describe the conditions of the system. In the case of a one-dimensional map or a single differential equation, the state is a single 
number. In the case of a two-dimensional map or a system of two coupled ordinary differential equations, it is a pair of numbers. The 
state is essentially the initial information needed for the dynamical system model to operate and respond unambiguously with an 
orbit. In partial differential equations models, the state space may be infinite-dimensional. In such a case, infinitely many numbers 
are needed to specify the current state. For a vibrating string, modeled by a partial differential equation called the wave equation, 
the state is the entire real-valued function describing the shape of the string.

You are already familiar with systems that need more than one number to specify the current condition of the system. For thesystem 
consisting of a projectile falling under Newton’s laws of motion, the state of the system at a given time can be specified fully by 
six numbers. If we know the position  p  (x, y, z) and velocity  v  (v x , v y , v z ) of the projectile at time t  0, then the state 
at any future time t is uniquely determined as

x(t)  x(0)  v x (0)t

y(t)  y(0)  v y (0)t

z(t)  z(0)  v z (0)t  gt 2 ̸ 2

v x (t)  v x (0)

v y (t)  v y (0)

v z (t)  v z (0)  gt,

(2.5)

where the constant g represents the acceleration toward earth due to gravity. (In meters-kilograms-seconds units, g  9.8 m/sec 2 .) 
Formula (2.5) is valid as long as the projectile is aloft. The assumptions built into this model are that gravity alone is the only 
force acting on the projectile, and that the force of gravity is constant. The formula is again derived from a differential equation 
due to Newton, this time the law of motion

F  ma.

(2.6)

Here the gravitational force is

GMm F , r2 

(2.7)

where M is the mass of the earth, m is the mass of the projectile, G is the gravitational constant, and r is the distance from the 
center of the earth to the projectile. Since r is constant to good approximation as long as the projectile is near the surface of the 
earth, g can be calculated 
as GM ̸ r 2 .

The position of the falling projectile evolves according to a set of six equations (2.5). The identification of a projectile’s motion 
as a system with a six-dimensional state is one of the great achievements of Newtonian physics. However, it is rare to find a 
dynamical system that has an explicit formula like (2.5) that describes the evolution of the state’s future. More often we know a 
formula that describes the new state in terms of the previous and/or current state.

A map is a formula that describes the new state in terms of the previous state. We studied many such examples in Chapter 1. A 
differential equation is a formula for the instantaneous rate of change of the current state, in terms of the current state. An 
example of a differential equation model is the motion of a projectile far from the earth’s surface. For example, a satellite falling 
to earth must follow the gravitational equation (2.7) but with a nonconstant distance r. That makes the acceleration a function of 
the position of the satellite, rendering equations (2.5) invalid.

A system consisting of two orbiting masses interacting through gravitational acceleration can be expressed as a differential 
equation. Using Newton’s law of motion (2.6) and the gravitation formula (2.7), the motions of the two masses can be derived as a 
function of time as in (2.5). We say that such a system is “analytically solvable”. The resulting formulas show that the masses 
follow elliptical orbits around the combined center of mass of the two bodies.

On the other hand, a system of three or more masses interacting exclusively through gravitational acceleration is not analytically 
solvable. The socalled three-body problem, for example, has an 18-dimensional state space. To solve the equations needed to advance 
the dynamics, one must know the three positions and three velocities of the point masses, a total of six three-dimensional vectors, 
or 18 numbers. Although there are no exact formulas of type (2.5) in this case, one can use computational methods to approximate the 
solution of the differential equations resulting from Newton’s laws of motion to get an idea of the complicated behavior that 
results.

At one time it was not known that there are no such exact formulas. In 1889, to commemorate the 60th birthday of King Oscar II of 
Sweden and Norway, a contest was held to produce the best research in celestial mechanics pertaining to the stability of the solar 
system, a particularly relevant n-body problem. The winner was declared to be Henri Poincar´e, a professor at the University of 
Paris.

Poincar´e submitted an entry full of seminal ideas. In order to make progress on the problem, he made two simplifying assumptions. 
First, he assumed that there are three bodies all moving in a plane. Second, he assumed that two of
the bodies were massive and that the third had negligible mass in comparison, so small that it did not affect the motion of the other 
two. We can imagine two stars and a small asteroid. In general, the two large stars would travel in ellipses, but Poincar´e made 
another assumption, that the initial conditions were chosen such that the two moved in circles, at constant speed, circling about 
their combined center of mass.

It is simplest to observe the trajectory of the asteroid in the rotating coordinate system in which the two stars are stationary. 
Imagine looking down on the plane in which they are moving, rotating yourself with them so that the two appear fixed in position. 
Figure 2.1 shows a typical path of the asteroid. The horizontal line segment in the center represents the (stationary) distance 
between the two larger bodies: The large one is at the left end of the segment and the medium one is at the right end. The tiny body 
moves back and forth between the two larger bodies in a seemingly unpredictable manner for a long time. The asteroid gains speed as 
it is ejected toward the right with sufficient momentum so

Figure 2.1 A trajectory of a tiny mass in the three-body problem.

Two larger bodies are in circular motion around one another. This view is of a rotating coordinate system in which the two larger 
bodies lie at the left and right ends of the horizontal line segment. The tiny mass is eventually ejected toward the right. Other 
trajectories starting close to one of the bodies can be forever trapped.


that it never returns. Other pictures, with different initial conditions, could be shown in which the asteroid remains forever close 
to one of the stars.

This three-body system is called the “planar restricted three-body problem”, but we will refer to it as the three-body problem. 
Poincar´e’s method of analysis was based on the fact that the motion of the small mass could be studied, in a rather nonobvious 
manner, by studying the orbit of a plane map (a function from  2 to  2 ). He discovered the crucial ideas of “stable and unstable 
manifolds”, which are special curves in the plane (see Section 2.6 for definitions).

On the basis of his entry Poincar´e was declared the winner. However, he did not fully understand the nature of the stable and 
unstable manifolds at that time. These manifolds can cross each other, in so-called homoclinic points. Poincar´e was confused by 
these points. Either he thought they didn’t exist or he didn’t understand what happened when they did cross. This error made his 
general conclusions about the nature of the trajectories totally wrong. His error was detected after he had been declared the winner 
but before his entry was published. He eventually realized that the existence of homoclinic points implied that there was incredibly 
complicated motion near those points, behavior we now call “chaotic”.

Poincar´e worked feverishly to revise his winning entry. The article “Sur les ´equations de la dynamique et le probl`eme des trois 
corps” (on the equations of dynamics and the three-body problem) was published in 1890. In this 270-page work, Poincar´e established 
convincingly that due to the possibility of homoclinic crossings, no general exact formula exists, beyond Newton’s differential 
equations arising from (2.6) and (2.7), for making predictions of the positions of the three bodies in the future.

Poincar´e succeeded by introducing qualitative methods into an area of study that had long been dominated by highly refined 
quantitative methods. The quantitative methods essentially involved developing infinite series expansions for the positions and 
velocities of the gravitational bodies. These series expansions were known to be problematic for representing near-collisions of the 
bodies. Poincar´e was able to show through his geometric reasoning that these infinite series expansions could not converge in 
general.

One of Poincar´e’s most important innovations was a simplified way to look at complicated continuous trajectories, such as those 
resulting from differential equations. Instead of studying the entire trajectory, he found that much of the important information was 
encoded in the points in which the trajectory passed through a two-dimensional plane. The order of these intersection points defines 
a plane map. Figure 2.2 shows a schematic view of a trajectory C. The plane S is defined by x 3  constant. Each time the trajectory C 
pierces S in a downward

Figure 2.2 A Poincar´e map derived from a differential equation.

The map G is defined to encode the downward piercings of the plane S by the solution curve C of the differential equation so that 
G(A)  B, and so on.

direction, as at points A and B in Figure 2.2, we record the point of piercing on the plane S. We can label these points by the 
coordinates (x 1 , x 2 ). Let A represent the kth downward piercing of the plane and B the (k  1)th downward piercing. The Poincar´e 
map is the two-dimensional map G such that G(A)  B. The Poincar´e map is similar in principle to the time-T map we considered above, 
though different in detail. While the time-T map is stroboscopic (it logs the value of a variable at equal time intervals), the 
Poincar´e map records plane piercings, which need not be equally spaced in time. Although Figure 2.2 shows a plane, more general 
surfaces can be used. The plane or surface is called a surface of section.

Given A, the differential equations can be solved with A as an initial value, and the solution followed until the next downward 
piercing at B. Thus A uniquely determines B. This ensures that the map G is well-defined. This map can be iterated to find all 
subsequent piercings of S. In general, the Poincar´e map technique reduces a k-dimensional, continuous-time dynamical system to a (k  
1)-dimensional map. Much of the dynamical behavior of the trajectory C is present in the two-dimensional map G. For example, the 
trajectory C is periodic (forms a closed curve, which repeats the same dynamics forever) if and only if the plane map G has a 
periodic orbit.

Now we can explain how the differential equations of the three-body problem shown in Figure 2.1 led Poincar´e to a plane map. In our 
rotating coordinate system (where the two stars are fixed at either end of the line segment) the differential equations involve the 
position (x, y) and 
velocity (˙x, y ˙ ) of the asteroid. These four numbers determine the current state of the asteroid.

There is a constant of the motion, called the Hamiltonian. For this problem it is a function in the four variables that is constant 
with respect to time. It is like total energy of the asteroid, which never changes. The following fact can be shown regarding this 
Hamiltonian: for y  0 and for any particular x and x ˙ , the Hamiltonian reduces to y ˙ 2  C, where C 
 0. Thus when the vertical component of the position of the asteroid satisfies y  0, the vertical velocity component of the asteroid 
is restricted to the two possible values  √ C.

Because of this fact, it makes sense to consider the Poincar´e map, using upward piercings of the surface of section y  0. The 
variable y is the vertical component of the position, so this corresponds to an upward crossing of the horizontal line segment in 
Figure 2.1. There are two “branches” of the two-dimensional surface y  0, corresponding to the two possible values of y ˙ mentioned 
above. (The two dimensions correspond to independent choices of the numbers x and x ˙ .) We choose one branch, say the one that 
corresponds to the positive value of y ˙ , for our surface of section. What we actually do is follow the solution of the differential 
equation, computing x, y, x ˙ , y ˙ as we go, and at the instant when y goes from negative to positive, we check the current y ˙ from 
the differential equation. If y ˙  0, then an upward crossing of the surface has occurred, and x, x ˙ are recorded. This defines a 
Poincar´e map.

Starting the system at a particular value of (x, x ˙ ), where y is zero and is moving from negative to positive, signalled by y ˙  0, 
we get the image F(x, x ˙ ) by recording the new (x, x ˙ ) the next time this occurs. The Poincar´e map F is a two-dimensional map. 
What Poincar´e realized should now be clear to us. Even this restricted version of the full three-body problem contains much of the 
complicated behavior possible in two-dimensional maps, including chaotic dynamics caused by homoclinic crossings, shown in Figure 
2.24. Understanding these complications will lead us to the study of stable and unstable manifolds, in Section 2.6 at first, and then 
in more detail in Chapter 10.

Work in the twentieth century has continued to reflect the philosophy that much of the chaotic phenomena present in differential 
equations can be approached, through reduction by time-T maps and Poincar´e maps, by studying discrete-time dynamics. As you can 
gather from the three-body example, Poincar´e maps are seldom simple to evaluate, even by computer. The French astronomer M. H´enon 
showed in 1975 that much of the interesting phenomena present in Poincar´e maps of differential equations can be found as well in a 
two-dimensional quadratic map, which is much easier to simulate on a computer. The version of H´enon’s map that we will study is

f(x, y)  (a  x 2  by, x).

(2.8)

Note that the map has two inputs, x, y, and two outputs, the new x, y. The new y is just the old x, but the new x is a nonlinear 
function of the old x and y. The letters a and b represent parameters that are held fixed as the map is iterated.

The nonlinear term x 2 in (2.8) is just about as unobtrusive as could be achieved. H´enon’s remarkable discovery is that this “barely 
nonlinear” map displays an impressive breadth of complex phenomena. In its way, the H´enon map is to two-dimensional dynamics what 
the logistic map G(x)  4x(1  x) is to one-dimensional dynamics, and it continues to be a catalyst for deeper understanding of 
nonlinear systems.

For now, set a  1.28 and b  0.3. (Here we diverge from H´enon, whose most famous example has b  0.3 instead.) If we start with the 
initial condition (x, y)  (0, 0) and iterate this map, we find that the orbit moves toward a periodtwo sink. Figure 2.3(a) shows an 
analysis of the results of iteration with general initial values. The picture was made by starting with a 700 700 grid of points

Figure 2.3 A square of initial conditions for the H´enon map with b  0.3. Initial values whose trajectories diverge to infinity upon 
repeated iteration are colored black. The crosses show the location of a period-two sink, which attracts the white initial values. 
(a) For a  1.28, the boundary of the basin is a smooth curve. (b) For a  1.4, the boundary is “fractal” .


in [2.5, 2.5] [2.5, 2.5] as initial values. The map (2.8) was iterated by computer for each initial value until the orbit either 
converges to the period-two sink, or diverges toward infinity. Points in black represent initial conditions whose orbits diverge to 
infinity, and points in white represent initial values whose orbits converge to the period-two orbit. The black points are said to 
belong to the basin of infinity, and the white points to the basin of the period-two sink. The boundary of the basin consists of a 
smooth curve, which moves in and out of the rectangle of initial values shown here.

EXERCISE T2.1

Check that the period-two orbit of the H´enon map (2.8) with a  1.28 and b  0.3 is approximately  (0.7618, 0.5382), (0.5382, 0.7618)  
. We will see how to find these points in Section 2.5.

COMPUTER EXPERIMENT 2.1

Write a program to iterate the H´enon map (2.8). Set a  1.28 and b  0.3 as in Figure 2.3(a). Using the initial condition (x, y)  (0, 
0), create the periodtwo orbit, and view it either by printing a list of numbers or by plotting (x, y) points. Change a to 1.2 and 
repeat. How does the second orbit differ from the first? Find as accurately as possible the value of a between 1.2 and 1.28 at which 
the orbit behavior changes from the first type to the second.

If we instead set a  1.4, with b  0.3, and repeat the process, we see quite a different picture. First, the points of the period-two 
sink have distanced themselves a bit from one another. More interesting is that the boundary of the basin is no longer a smooth 
curve, but is in a sense infinitely complicated. This is fractal structure, which we shall discuss in detail in Chapter 4.

Next we want to show how a two-dimensional map can be derived from a differential equations model of a pendulum. Figure 2.4 shows a 
pendulum swinging under the influence of gravity. We will assume that the pendulum is free to swing through 360 degrees. Denote by 
the angle of the pendulum with respect to the vertical, so that  0 corresponds to straight down. Therefore and  2  should be 
considered the same position of the pendulum.

Figure 2.4 The pendulum under gravitational acceleration.

The force of gravity causes the pendulum bob to accelerate in a direction perpendicular to the rod. Here denotes the angle of 
displacement from the downward position.

We will use Newton’s law of motion F  ma to find the pendulum equation. The motion of the pendulum bob is constrained to be along a 
circle of radius l, where l is the length of the pendulum rod. If is measured in radians, then the component of acceleration tangent 
to the circle is l ¨ , because the component of position tangent to the circle is l . The component of force along the direction of 
motion is mg sin . It is a restoring force, meaning that it is directed in the opposite direction from the displacement of the 
variable . We will denote the first and second time derivatives of by ˙ (the angular velocity) and ¨ (the angular acceleration) in 
what follows. The differential equation governing the frictionless pendulum is therefore

ml ¨  F  mg sin ,

(2.9)

according to Newton’s law of motion.

From this equation we see that the pendulum requires a two-dimensional state space. Since the differential equation is second order, 
the two initial values (0) and ˙ (0) at time t  0 are needed to specify the solution of the equation after time t  0. It is not 
enough to specify (0) alone. Knowing (0)  0 means that the pendulum is in the straight down configuration at time t  0, but we can’t 
predict what will happen next without knowing the angular velocity at that time. For example, if the angle is 0, so that the pendulum 
is hanging straight down, we need to know whether the angular velocity ˙ is positive or negative to tell whether the bob is moving 
right or left. The 
same can be said for knowing the angular velocity alone. Specifying and ˙ together at time t  0 will uniquely specify (t). Thus the 
state space is two-dimensional, and the state consists of the two numbers ( , ˙ ).

To simplify matters we will use a pendulum of length l  g, and to the resulting equation ¨   sin we add the damping term c ˙ , 
corresponding to friction at the pivot, and a periodic term 
 sin t which is an external force constantly providing energy to the pendulum. The resulting equation, which we

Figure 2.5 Basins of three coexisting attracting fixed points.

Parameters for the forced damped pendulum are c  0.2, 
  1.66. The basins are shown in black, gray, and white. Each initial value ( , ˙ ) is plotted according to the sink to which it is 
attracted. Since the horizontal axis denotes angle in radians, the right and left edge of the picture could be glued together, 
creating a cylinder. The rectangle shown is magnified in figure 2.6.


call the forced damped pendulum model, is


sin t.

(2.10)

This differential equation includes friction (the friction constant is c) and the force of gravity, which pulls the pendulum bob 
down, as well as the sinusoidal force 
 sin t, which accelerates the bob first clockwise and then counterclockwise, continuing in a periodic way. This periodic forcing 
guarantees that the pendulum will keep swinging, provided 
 is nonzero. The term 
 sin t has period 2  .

Because of the periodic forcing, if (t) is a solution of (2.10), then so is (t  2  ), and in fact so is (t  2  N) for each integer N. 
Assume that (t) is a solution of the differential equation (2.10), and define u(t)  (t  2  ). Why is u(t) also a solution of (2.10)? 
Note first that u ˙ (t)  ˙ (t  2  ) and ¨u(t)  ¨ (t  2  ). Second, since (t) is assumed to be a solution for all t, (2.10) implies

¨ (t  2  )  c ˙ (t  2  )  sin (t  2  )  
 sin(t  2  ).

(2.11)

Since the nonhomogeneous term of the differential equation is periodic with period 2  , sin t  sin(t  2  ), it follows that

¨u(t)  c˙u(t)  sin u(t)  
 sin t.

(2.12)

(a)

(b)

Figure 2.6 Detail views of the pendulum basin.

(a) Magnification of the rectangle shown in Figure 2.5. (b) Magnification of the rectangle shown in (a).


Note that this argument depends on the fact that u(t) is a translation of (t) by exactly a multiple of 2  . The argument fails if we 
choose, for example, u(t)  (t   ̸ 2) (assuming 
  0).


We conclude from this fact that the time-2  map of the forced damped ˙1  pendulum is well-defined. If ( 1 , ) is the result of 
starting with initial conditions ˙0  ( 0 , ) at time t  0 and following the differential equation for 2  time units, ˙ 1 ˙0  then ( 1 
, ) will also be the result of starting with initial conditions ( 0 , ) at time t  2  (or 4  , 6  , . . .) and following the 
differential equation for 2  time units. This fact allows us to study many of the important features by studying the time-2  map of 
the pendulum. Because the forcing is periodic with period 2  , the action of the differential equation is the same between 2N  and 
2(N  1)  for each integer N. Although the state equations for this system are differential equations, we can learn a lot of 
information about it by viewing snapshots taken each 2  time units.

When the pendulum is started at time t  0, its behavior will be determined by the initial values of and ˙ . The differential equation 
uniquely determines the values of and ˙ at later times, such as t  2  . If we write ( 0 , ˙ 0 ) for the ˙1  initial values and ( 1 , 
) as the values at time 2  , we can define the time-2  map F by

˙ 0 ˙1  F( 0 , )  ( 1 , ).

(2.13)

Just because we give the time-2  map a name does not mean that there is a simple formula for computing it. Analyzing the time-2  map 
is different from analyzing the H´enon map, in the sense that there is no simple expression for the former map. The differential 
equation must be solved from time 0 to time 2  in order to iterate the map. For this example, investigation must be carried out 
largely by computer.

Figure 2.5 shows the basins of three coexisting attractors for the time-2  map of the forced damped pendulum. Here we have set the 
forcing parameter 
  1.66 and the damping parameter c  0.2. The picture was made by solving the differential equation for an initial condition 
representing each pixel, and coloring the pixel white, gray, or black depending on which sink orbit attracts the orbit.

The three attractors are one fixed point and two period-two orbits. There are five other fixed points that are not attractors. This 
system displays both great simplicity, in that the stable behaviors (sinks) are periodic orbits of low period, and great complexity, 
in that the boundaries between the three basins are infinitely-layered, or fractal.

Figure 2.6 shows further detail of the pendulum basins. Part (a) zooms in on the rectangle in Figure 2.5; part (b) is a further 
magnification. Note that the level of complication does not decrease upon magnification. The fractal structure continues on finer and 
finer scales. Color Plates 3–10 show an alternate setting of parameters for which four basins exist.

For other sets of parameter values, there are apparently no sinks for the forced damped pendulum, fixed or periodic. Figure 2.7 shows 
a long orbit of the pendulum with parameter settings c  0.05, 
  2.5. One-half million points are shown. If these points were erased, and the next one-half million points were plotted, the picture 
would look the same. There are many fixed points and periodic orbits that coexist with this orbit. Some of them are shown in Figure 
2.23(a).


Figure 2.7 A single orbit of the forced damped pendulum with c  0.05,   2.5.

Different initial values yield essentially the same pattern, unless the initial value is an unstable periodic orbit, of which there 
are several (see Figure 2.23).


2.2 SINKS , SOURCES , AND SADDLES

We introduced the term “sink” in our discussion of one-dimensional maps to refer to a fixed point or periodic orbit that attracts an  
-neighborhood of initial values. A source is a fixed point that repels a neighborhood. These definitions make sense in 
higher-dimensional state spaces without alteration. In the plane, for example, the neighborhoods in question are disks (interiors of 
circles).

Definition 2.1  x m 2 . Let p  (p 1 , p 2 , . . . , p m )   m , and let  be a positive number. The -neighborhood N  (p) is the set  v   
m : |v  p|    
, the set of points within Euclidean distance  of p. We sometimes call N  (p) an -disk centered at p.

Definition 2.2 Let f be a map on  m and let p in  m be a fixed point, that is, f(p)  p. If there is an   0 such that for all v in the  
-neighborhood N  (p), lim k→ f k (v)  p, then p is a sink or attracting fixed point. If there is an  -neighborhood N  (p) such that 
each v in N  (p) except for p itself eventually maps outside of N  (p), then p is a source or repeller.

Figure 2.8 shows schematic views of a sink and a source for a two-dimensional map, along with a typical disk neighborhood and its 
image under the map. Along with the sink and source, a new type of fixed point is shown in Figure 2.8(c), which cannot occur in a 
one-dimensional state space. This type of fixed point, which we will call a saddle, has at least one attracting direction and at 
least one repelling direction. A saddle exhibits sensitive dependence on initial conditions, because of the neighboring initial 
conditions that escape along the repelling direction.

EXAMPLE 2.3

Consider the two-dimensional map

f(x, y)  (x 2  0.4y, x).

(2.14)

This is a version of the H´enon map considered earlier in this chapter, with the parameters set at a  0 and b  0.4.


Figure 2.8 Local dynamics near a fixed point.

The origin is (a) a sink, (b) a source, and (c) a saddle. Shown is a disk N and its iterate under the map f.

EXERCISE T2.2

Show that the map in (2.14) has exactly two fixed points, (0, 0) and (0.6, 0.6).

Figure 2.9 shows the two fixed points. Around each is drawn a small disk N of radius 0.3. Also shown are the images f(N) and f 2 (N) 
of each disk. The fixed point (0, 0) is a sink, and the fixed point (0.6, 0.6) is a saddle. Each time the map is iterated, the disks 
shrink to 40% of their previous size. Therefore f(N) is 40% the size of N, and f 2 (N) is 16% the size of N. We will explain the 
origin of these numbers in Remark 2.15.

Although saddles, as well as sources, are unstable fixed points (they are sensitive to initial conditions), they play surprising 
roles in the dynamics. In Figure 2.10(a), the basin of the sink (0, 0) is shown in white. The entire square is the box [2, 2] [2, 2], 
and the sink is the cross at the center. Not all of the white basin is shown: it has infinite area. The points in black diverge under 
iteration by f; they are in the basin of infinity. You may wonder about the final disposition of the points along the boundary 
between the two basins. Do they go in or out? The answer is: neither. In Figure 2.10(b), the set of points that converge to the 
saddle (0.6, 0.6) is plotted, along with the saddle denoted by the cross. Although not an attractor, the saddle evidently plays a 
decisive role in determining which points go to which basin.

Figure 2.9 Local dynamics near fixed points of the H´enon map.

The crosses mark two fixed points of the H´enon map f with a  0, b  0.4, in the square [1.5, 0.5] [1.5, 0.5]. Around each fixed point 
a circle is drawn along with its two forward images under f. On the left is a saddle: the images of the disk are becoming 
increasingly long and thin. On the right the images are shrinking, signifying a sink.

Attractors, as well as basins, can be more complicated than those shown in Figure 2.10. Consider the H´enon map (2.8) with a  2 and b  
0.3. In Figure 2.11 the dark area is again the basin of infinity, while the white set is the basin for the two-piece attractor that 
looks like two hairpin curves.

Our goal in the next few sections is to find ways of identifying sinks, sources and saddles from the defining equations of the map. 
In Chapter 1 we found that the key to deciding the stability of a fixed point is the derivative at the point. Since the derivative 
determines the tangent line, or best linear approximation near the point, it determines the amount of shrinking/stretching in the 
vicinity of the point. The same mechanism is operating in higher dimensions. The action of the dynamics in the vicinity of a fixed 
point is governed by the best linear approximation to the map at the point. This best linear approximation is given by the Jacobian 
matrix, a matrix of partial derivatives calculated from the map. We will define the Jacobian matrix in Section 2.5. To find out what 
it can tell us, we need to fully understand linear maps first. For linear maps, the Jacobian matrix is equal to the map itself.

Figure 2.10 Basins of attraction for the H´enon map with a  0, b  0.4.

(a) The cross marks the fixed point (0, 0). The basin of the fixed point (0, 0) is shown in white; the points in black diverge to 
infinity. (b) The initial conditions that are on the boundary between the white and black don’t converge to (0, 0) or infinity; 
instead they converge to the saddle (0.6, 0.6), marked with a cross. This set of boundary points is the stable manifold of the saddle 
(to be discussed in Section 2.6).

Figure 2.11 Attractors for the H´enon map with a  2, b  0.3.

Initial values in the white region are attracted to the hairpin attractor inside the white region. On each iteration, the points on 
one piece of the attractor map to the other piece. Orbits from initial values in the black region diverge to infinity.


2.3 LINEAR MAPS

The linear maps on  2 are those of the particularly simple form v A is a 2 2 matrix:

Av, where

x a 11 a 12 x a11 x  a12 y A   . y a 21 a 22 y a21 x  a22 y ( ) ( )( ) ( )

(2.15)

Definition 2.4 A map A(v) from  m to  m is linear if for each a, b  , and v, w   m , A(av  bw)  aA(v)  bA(w). Equivalently, a linear 
map A(v) can be represented as multiplication by an m m matrix.

Every linear map has a fixed point at the origin. This is analogous to the one-dimensional linear map f(x)  ax. The stability of the 
fixed point will be investigated the same way as in Chapter 1. If all of the points in a neighborhood of the fixed point (0, 0) 
approach the fixed point when iterated by the map, we consider the fixed point to be an attractor.

------------------------------------------------------------------------------------

In some cases the dynamics for a two-dimensional map resemble one-dimensional dynamics. 
Recall that A is an eigenvalue of the matrix A if there is a nonzero vector v such that

 v.vector v such that

 , we can write down a special trajectory





We will begin by looking at three important examples of linear maps on  2 . In fact, the three different types of 2 2 matrices we 
will encounter will be more than just good examples, they will be all possible examples, up to change of coordinates.

62 2.3 L I N E A R M A P S

E XAM PLE 2.5

[Distinct real eigenvalues.] Let v  (x, y) denote a two-dimensional vector, and let A(v) be the map on  2 defined by

A(x, y)  (ax, by).

Each input is a two-dimensional vector; so is each output. Any linear map can be represented by multiplication by a matrix, and 
following tradition we use A also to represent the matrix. Thus

A(v)  Av 

a 0 (

0 b

x y )( )

.

(2.16)

The eigenvalues of the matrix A are a and b, with associated eigenvectors (1, 0) and (0, 1), respectively. For the purposes of this 
example, we will assume that they are not equal, although most of what we say now will not depend on that fact. Part of the 
importance of this example comes from the fact that a large class of linear maps can be expressed in the form (2.16), if the right 
coordinate system is used. For example, it is shown in Appendix A that any 2 2 matrix with distinct real eigenvalues takes this form 
when its eigenvectors are used to form the basis vectors of the coordinate system.

For the map in Example 2.5, the result of iterating the map n times is represented by the matrix

a n 0 An  . 0 bn  ( )

(2.17)

The unit disk is mapped into an ellipse with semi-major axes of length |a| n along the x-axis and |b| n along the y-axis. An epsilon 
disk N  (0, 0) would become an ellipse with axes of length |a| n and |b| n . For example, suppose that a and b are smaller than 1 in 
absolute value. Then this ellipse shrinks toward the origin as n → , so (0, 0) is a sink for A. If |a|, |b|  1, then the origin is a 
source. On the other hand, if |a|  1  |b|, we see dynamical behavior that is not seen in one-dimensional maps. As n is increased, the 
ellipse grows in the x-direction and shrinks in the y-direction, essentially growing to look like the x-axis as n → . In Figure 
2.12, we plot the unit disk and its two iterates under A where we set a  2 and b  1 ̸ 2. In this case, the origin is neither a sink 
nor a source. If the ellipses formed by successive iterates of the map grow without bound along one direction and shrink to zero 
along another, we will call the origin a saddle.

63 T WO -D I M E N S I O N A L M A P S

y

2

-4

4

x

-2

Figure 2.12 The unit disk and two images of the unit disk under a linear map. The origin is a saddle fixed point.

Points act as if they were moving along the surface of a saddle under the influence of gravity. A cowboy who spills coffee on his 
saddle will see it run toward the center along the front-to-back axis of the saddle (the y-axis in Figure 2.13) and run away from the 
center along the side-to-side axis (the x-axis in Figure 2.13). Presumably, a drop situated at the exact center of the saddle would 
stay there (our assumption is that the horse is not moving).

We see the same behavior for the iteration of points in Figure 2.14, which illustrates the linear map represented by the matrix

A

2 0 (

0

)

.

(2.18)

0.5

y

x

Figure 2.13 Dynamics near a saddle point.

Points in the vicinity of a saddle fixed point (here the origin in the xy-plane) move as if responding to the influence of gravity on 
a saddle.

64 2.3 L I N E A R M A P S

y

x

Figure 2.14 Saddle dynamics.

Successive images of points near a saddle fixed point are shown.

A typical point (x 0 , y 0 ) maps to (2x 0 , 1 2 y 0 ), and then to (4x 0 , 1 4 y 0 , ), and so on. Notice that the product of the x- 
and y- coordinates is the constant quantity x 0 y 0 , so that orbits shown in Figure 2.14 traverse the hyperbola xy  constant  x 0 y 
0 . More generally, for a linear map A on  2 , the origin is a saddle if and only if iteration of the unit disk results in ellipses 
whose two axis lengths converge to zero and infinity, respectively.

A simplification can be made when analyzing small neighborhoods under v linear maps. Because linearity implies A(v)  |v|A( |v| ), the 
image of a vector v can be found by mapping the unit vector in the direction of v, followed by scalar multiplication by the magnitude 
|v|. The effect of the map on a small disk neighborhood of the origin is just a scaled-down version of the effect on the unit disk N  
N 1 (0, 0)   v : |v|  1  . As a result we will often restrict our attention to the effect of the matrix on the unit disk. For 
example, the image of the unit disk centered at the origin under multiplication by any matrix is a filled ellipse centered at the 
origin. If the radius of the disk is r instead of 1, the resulting ellipse will also be changed precisely by a factor of r. (The 
semi-major axes will be changed by a factor of r.)

E XAM PLE 2.6

[Repeated eigenvalue.] For an example where the eigenvalues are not distinct, let

A

a 0 (

1 a

)

.

(2.19)

65 T WO -D I M E N S I O N A L M A P S

The eigenvalues of an upper triangular matrix are its diagonal entries, so the matrix has a repeated eigenvalue of a. Check that

a n An a n1 . 0 a ( )

(2.20)

Therefore the effect of A n on vectors is

x ax  ny A n a n1 . y ay ( ) ( )

(2.21)

✎

E XERCISE T2.3

(a) Verify equation (2.20). (b) Use equation (2.21) to show that the fixed point (0, 0) is a sink if |a|  1 and a source if |a|  1.

E XAM PLE 2.7

[Complex eigenvalues.] Let

A

a b (

b a

)

.

(2.22)

This matrix has no real eigenvalues. The eigenvalues of this matrix are a  bi and a  bi, where i  √ 1. The corresponding 
eigenvectors are (1, i) and (1, i), respectively. Fortunately, this information can be interpreted in terms of real vectors. A more 
intuitive way to look at this matrix follows from multiplying and dividing by r  √ a 2  b 2 . Then

a ̸ r b ̸ r cos  sin A  r  √ a 2  b 2 . b r a r sin cos ( ̸ ̸ ) ( )

(2.23)

Here we used the fact that any pair of numbers c, s such that c 2  s 2  1 can be written as c  cos and s  sin for some angle . The 
angle can be identified as  arctan(b ̸ a). It is now clear that multiplication by this matrix rotates points about the origin by an 
angle , and multiplies the distances by √ a 2  b 2 . Therefore it is a combination of a rotation and a dilation.

✎

E XERCISE T2.4

Verify that multiplication by A rotates a vector by arctan(b ̸ a) and stretches

by a factor of √ a 2  b 2 .

66 2.4 C O O R D I N AT E C H A N G E S

In summary, the effect of multiplication by A on the length of a vector is contraction/expansion by a factor of √ a 2  b 2 . It 
follows that the stability result is the same as in the previous two cases: If the magnitude of the eigenvalues is less than 1, the 
origin is a sink; if greater than 1, a source.

2.4 C OORDINATE C HANGES

Now that we have some experience with iterating linear maps, we return to the fundamental issue of how a matrix represents a linear 
map. Changes of coordinates can simplify stability calculations for higher-dimensional maps.

A vector in  m can be represented in many different ways, depending on the coordinate system chosen. Choosing a coordinate system is 
equivalent to choosing a basis of  m ; the coordinates of a vector are simply the coefficients which express the vector in that 
basis. Changing the basis of  m requires changing the matrix representing the linear map A(v). In particular, let S be a square 
matrix whose columns are the new basis vectors. Then the matrix S 1 AS represents the linear map in the new basis. A matrix of form S 
1 AS, where S is a nonsingular matrix, is similar to A.

Similar matrices have the same set of eigenvalues and the same determinant. The determinant det(A)  a 11 a 22  a 12 a 21 is a measure 
 c. It stands to reason that area transformation should be independent of the choice of coordinates. See Appendix A for justification 
of these statements and for a thorough discussion of changes of coordinates.

Matrices that are similar have the same dynamical properties when viewed as maps, since they only differ by the coordinate system 
used to view them. For example, the property that a small neighborhood of the fixed point origin is attracted to the origin is 
independent of the choice of coordinates. If (0, 0) is a sink under A, it remains so under S 1 AS. This puts us in position to 
analyze the dynamics of all linear maps on  2 , because of the following fact: All 2 2 matrices are similar to one of Examples 2.5, 
2.6, 2.7. See Appendix A for a proof of this fact.

Since similar matrices have identical eigenvalues, deciding the stability of the origin for a linear map A(v) is as simple as 
computing the eigenvalues of a matrix representation A. For example, if the eigenvalues a and b of A are real and distinct, then A is 
similar to the matrix

a 0 A2  . 0 b ( )

(2.24)

67 T WO -D I M E N S I O N A L M A P S

Therefore the map has this matrix representation in some coordinate system. Referring to Example 2.5, we see that the origin is a 
sink if |a|, |b|  1 and a source if |a|, |b|  1. The same analysis works for matrices with repeated eigenvalues, or a pair of complex 
eigenvalues. Summing up, we have proved the m  2 version of the following theorem.

Theorem 2.8 Let A(v) be a linear map on  m , which is represented by the matrix A (in some coordinate system). Then

1. The origin is a sink if all eigenvalues of A are smaller than one in absolute value;

2. The origin is a source if all eigenvalues of A are larger than one in absolute value.

In dimensions two and greater, we must also consider linear maps of mixed stability, i.e., those for which the origin is a saddle.

Definition 2.9 Let A be a linear map on  m . We say A is hyperbolic if A has no eigenvalues of absolute value one. If a hyperbolic 
map A has at least one eigenvalue of absolute value greater than one and at least one eigenvalue of absolute value smaller than one, 
then the origin is called a saddle.

Thus there are three types of hyperbolic maps: ones for which the origin is a sink, ones for which the origin is a source, and ones 
for which the origin is a saddle. Hyperbolic linear maps are important objects of study because they have well-defined expanding and 
contracting directions.

2.5 N ONLINEAR M APS AND THE

J ACOBIAN M ATRIX

So far we have discussed linear maps, which always have a fixed point at the origin. We now want to discuss nonlinear maps, and in 
particular how to determine the stability of fixed points.

Our treatment of stability in Chapter 1 is relevant to this case. Theorem 1.5 showed that whether a fixed point of a one-dimensional 
nonlinear map is a sink or source depends on its “linearization”, or linear part, near the fixed point. In the one-dimensional case 
the linearization is given by the derivative at the fixed point. If p is a fixed point and h is a small number, then the change in 
the

68 2.5 N O N L I N E A R M A P S A N D T H E J AC O B I A N M AT R I X

output of the map at p  h, compared to the output at p, is well approximated by the linear map L(h)  Kh, where K is the constant 
number f′ (p). In other words,

f(p  h)  f(p)  hf′ (p).

(2.25)

Our proof of Theorem 1.5 was based on the fact that the error in this approximation was of size proportional to h 2 . This can be 
made as small as desired by restricting attention to sufficiently small h. If |f′ (p)|  1, the fixed point p is a sink, and if |f′ 
(p)|  1, it is a source. The situation is very similar for nonlinear maps in higher dimensions. The place of the derivative in the 
above discussion is taken by a matrix.

Definition 2.10 Let f  (f 1 , f 2 , . . . , f m ) be a map on  m , and let p   m . The Jacobian matrix of f at p, denoted Df(p), is 
the matrix

 (p) ⎝  x 1  x m ⎠

of partial derivatives evaluated at p.

Given a vector p and a small vector h, the increment in f due to h is approximated by the Jacobian matrix times the vector h:

 h, (2.26) where again the error in the approximation is proportional to |h| 2 for small h. If we assume that f(p)  p, then for a 
 h in output.

As long as this deviation remains small (so that |h| 2 is negligible and our approximation is valid), the action of the map near p is 
essentially the same as the linear map h ↦ → Ah, where A  Df(p), with fixed point h  0. Small disk neighborhoods centered at h  0 
(corresponding to disks around p) map to regions approximated by ellipses whose axes are determined by A. In that case, we can appeal 
to Theorem 2.8 for information about linear stability for higher-dimensional maps in order to understand the nonlinear case.

The following theorem is an extension of Theorems 1.5 and 2.8 to higher dimensional nonlinear maps. It determines the stability of a 
map at a fixed point based on the Jacobian matrix at that point. The proof is omitted.

69 T WO -D I M E N S I O N A L M A P S

Theorem 2.11 Let f be a map on  m , and assume f(p)  p.

1. If the magnitude of each eigenvalue of Df(p) is less than 1, then p is a sink.

2. If the magnitude of each eigenvalue of Df(p) is greater than 1, then p is a source.

Just as linear maps of  m for m  1 can have some directions in which orbits diverge from 0 and some in which orbits converge to 0, so 
fixed points of nonlinear maps can attract points in some directions and repel points in others.

Definition 2.12 Let f be a map on  m , m 
 1. Assume that f(p)  p. Then the fixed point p is called hyperbolic if none of the eigenvalues of Df(p) has magnitude 1. If p is 
hyperbolic and if at least one eigenvalue of Df(p) has magnitude greater than 1 and at least one eigenvalue has magnitude less than 
1, then p is called a saddle. (For a periodic point of period k, replace f by f k .)

Saddles are unstable. If even one eigenvalue of Df(p) has magnitude greater than 1, then p is unstable in the sense previously 
described: Almost any perturbation of the orbit away from the fixed point will be magnified under iteration. In a small epsilon 
neighborhood of p, f behaves very much like a linear map with an eigenvalue that has magnitude greater than 1; that is, the orbits of 
most points near p diverge from p.

E XAM PLE 2.13

The H´enon map

f a,b (x, y)  (a  x 2  by, x),

(2.27)

where a and b are constants, has at most two fixed points. Setting a  0 and b  0.4, f has the two fixed points (0, 0) and (0.6, 0.6). 
The Jacobian matrix Df is

( Evaluated at (0, 0), the Jacobian matrix is

Df(x, y) 

2x 1

b 0

)

.

(2.28)

0 0.4 , 1 0 ( ) with eigenvalues  √ 0.4, approximately equal to 0.632 and 0.632. Evaluated at (0.6, 0.6), the Jacobian is

Df(0, 0) 

Df(0.6, 0.6) 

1.2 0.4 1 0 (

)

,

70 2.5 N O N L I N E A R M A P S A N D T H E J AC O B I A N M AT R I X

with eigenvalues approximately equal to 1.472 and 0.272. Thus (0, 0) is a sink and (0.6, 0.6) is a saddle.

For the parameter values a  0.43, b  0.4, there is a period-two orbit for the map. Check that  (0.7, 0.1), (0.1, 0.7)  is such an 
orbit. In order to check the stability of this orbit, we need to compute the Jacobian matrix of f 2 evaluated at (0.7, 0.1). Because 
 Df(x). We compute

 Df((0.7, 0.1)) 2(0.1) 0.4 2(0.7) 0.4  1 0 1 0 ( )(

)



0.12 0.08 1.4 0.4 (

)

.

The eigenvalues of this Jacobian matrix are approximately 0.26  0.30i, which are complex numbers of magnitude  0.4, so the period-two 
orbit is a sink.

Note that the same eigenvalues are obtained by evaluating

 Df((0.1, 0.7)), which means that stability is a property of the periodic orbit as a whole, not of the individual points of the 
orbit. This is true because the eigenvalues of a product AB of two matrices are identical to the eigenvalues of BA, as shown in the 
Appendix A. This result compares with (1.4) of Chapter 1.

Remark 2.14 For a map on  m , there is a more general statement of this fact. Assume there is a periodic orbit  p 1 , . . . , p k  of 
period k. By Lemma A.2 of Appendix A, the set of eigenvalues of a product of several matrices is unchanged under a cyclic permutation 
of the order of the product. Using the chain rule,

 Df(p 1 ).

(2.29)

The eigenvalues of the m m Jacobian matrix evaluated at p 1 , Df k (p 1 ), will determine the stability of the period-k orbit. But 
one should also be able to determine the stability by examining the eigenvalues of Df k (p r ), where p r is one of the other points 
in the periodic orbit. Applying the chain rule as above, we find that

 Df(p r ).

(2.30)

According to Lemma A.2, the eigenvalues of (2.29) and (2.30) are identical. This guarantees that the eigenvalues are shared by the 
periodic orbit, and can be

71 T WO -D I M E N S I O N A L M A P S

measured by multiplying together the k Jacobian matrices starting at any of the k points.

A more systematic study can be made of the fixed points and period-two points of the H´enon map. Let the parameters a and b be 
arbitrary. Then all fixed points satisfy

x  a  x 2  by

y  x, which is equivalent to the equation x  a  x2 

(2.31)

 bx, or

x 2  (1  b)x  a  0.

(2.32)

Using the quadratic formula, we see that fixed points exist as long as

4a  (1  b)2 

(2.33)

If (2.33) is satisfied, there are exactly two fixed points, whose x-coordinates are found from the quadratic formula and whose 
y-coordinate is the same as the x-coordinate.

To look for period-two points, set (x, y)  f 2 (x, y):

x  a  (a  x 2  by) 2  bx

y  a  x 2  by.

(2.34)

Solving the second equation for y and substituting into the first, we get an equation for the x-coordinate of a period-two point:

0  (x 2  a) 2  (1  b) 3 x  (1  b) 2 a

 (x 2  (1  b)x  a  (1  b) 2 )(x 2  (1  b)x  a).

(2.35)

We recognize the factor on the right from Equation (2.32): Zeros of it correspond to fixed points of f, which are also fixed points 
of f 2 . In fact, it was the knowledge that (2.32) must be a factor which was the trick that allowed us to write (2.35) in factored 
form. The period-two orbit is given by the zeros of the left factor, if they exist.

✎

E XERCISE T2.5

Prove that the H´enon map has a period-two orbit if and only if 4a

3(1  b) 2 .



72 2.5 N O N L I N E A R M A P S A N D T H E J AC O B I A N M AT R I X

y

a=1.6 1

a=.27

-1 a=-.09 a=.27

1

x

-1 a=1.6

Figure 2.15 Fixed points and period-two points for the H´enon map with b fixed at 0.4.

The solid line denotes the trails of the two fixed points as a moves from 0.09, where the two fixed points are created together, to 
1.6 where they have moved quite far apart. The fixed point that moves diagonally upward is attracting for 0.09  a  0.27; the other is 
a saddle. The dashed line follows the period-two orbit from its creation when a  0.27, at the site of the (previously) attracting 
fixed point, to a  1.6.

Figure 2.15 shows the fixed points and period-two points of the H´enon map for b  .4 and for various values of a. We understand why 
the fixed points lie along the diagonal line y  x, but why do the period-two orbits lie along a line, as shown in Figure 2.15?

✎

E XERCISE T2.6

(a) If (x 1 , y 1 ) and (x 2 , y 2 ) are the two fixed points of the H´enon map (2.27) with some fixed parameters a and b, show that 
x 1  y 1  x 2  y 2  0

and x 1  x 2  y 1  y 2  b  1.

(b) If  (x 1 , y 1 ), (x 2 , y 2 )  is the period-two orbit, show that x 1  y 1  x 2  y 2  x 1  x 2  y 1  y 2  1  b. In particular 
the period-two orbit lies along the line x  y  1  b, as seen in Figure 2.15.

Figure 2.16 shows a bifurcation diagram for the H´enon map for the case b  0.4. For each fixed value 0 a 1.25 along the horizontal 
axis, the xcoordinates of the attracting set are plotted vertically. The information in Figure

73 T WO -D I M E N S I O N A L M A P S

2.5

2.5

0

1.25

Figure 2.16 Bifurcation diagram for the H´enon map, b  0.4.

Each vertical slice shows the projection onto the x-axis of an attractor for the map for a fixed value of the parameter a.

2.15 is recapitulated here. At a  0.27, a period-doubling bifurcation occurs, when the fixed point loses stability and a period-two 
orbit is born. The periodtwo orbit is a sink until a  0.85, when it too doubles its period. In the next exercise, you will be asked 
to use the equations we developed here to verify some of these facts.

✎

E XERCISE T2.7

Set b  0.4.

(a) Prove that for 0.09  a  0.27, the H´enon map f has one sink

fixed point and one saddle fixed point.

(b) Find the largest magnitude eigenvalue of the Jacobian matrix at the first fixed point when a  0.27. Explain the loss of stability 
of the sink.

(c) Prove that for 0.27  a  0.85, f has a period-two sink.

(d) Find the largest magnitude eigenvalue of Df 2 , the Jacobian of f2  at the period-two orbit, when a  0.85.

For b  0.4 and a  0.85, the attractors of the H´enon map become more complex. When the period-two orbit becomes unstable, it is 
immediately replaced with an attracting period-four orbit, then a period-eight orbit, etc. Figure 2.17

74 2.5 N O N L I N E A R M A P S A N D T H E J AC O B I A N M AT R I X

Figure 2.17 Attractors for the H´enon map with b  0.4.

Each panel displays a single attracting orbit for a particular value of the parameter a. (a) a  0.9, period 4 sink. (b) a  0.988, 
period 16 sink. (c) a  1.0, four-piece attractor. (d) a  1.0293, period-ten sink. (e) a  1.045, two-piece attractor. The points of an 
orbit alternate between the pieces. (f) a  1.2, two pieces have merged to form one-piece attractor.

75 T WO -D I M E N S I O N A L M A P S

shows a number of these attractors. An example is the “period-ten window” at a  1.0293, barely detectable as a vertical white gap in 
Figure 2.16.

➮

COMPUTER

EXPERIMENT

2.2

Make a bifurcation diagram like Figure 2.16, but for b  0.3, and for 0 a 2.2. For each a, choose the initial point (0, 0) and 
calculate its orbit. Plot the x-coordinates of the orbit, starting with iterate 101 (to allow time for the orbit to approximately 
settle down to the attracting orbit). Questions to answer: Does the resulting bifurcation diagram depend on the choice of initial 
point? How is the picture different if the y-coordinates are plotted instead?

Periodic points are the key to many of the properties of a map. For example, trajectories often converge to a periodic sink. Periodic 
saddles and sources, on the other hand, do not attract open neighborhoods of initial values as sinks do, but are important in their 
own ways, as will be seen in later chapters.

Remark 2.15 The theme of this section has been the use of the Jacobian matrix for determining stability of periodic orbits of 
nonlinear maps, in the way that the map matrix itself is used for linear maps. There are other important uses for the Jacobian 
matrix. The magnitude of its determinant measures the transformation of areas for nonlinear maps, at least locally.

For example, consider the H´enon map (2.27). The determinant of the Jacobian matrix (2.28) is fixed at b for all v. For the case a  
0, b  0.4, the map f transforms area near each point v at the rate |det(Df(v))|  |  b|  0.4. Each plane region is transformed by f 
into a region that is 40% of its original size. The circle around each fixed point in Figure 2.9, for example, has forward images 
which are .4  40% and (.4) 2  .16  16%, respectively.

Most of the plane maps we will deal with are invertible, meaning that their inverses exist.

Definition 2.16

A map f on  m is one-to-one if f(v 1 )  f(v 2 ) implies

v1 v2 .

Recall that functions are well-defined by definition, i.e. v 1  v 2 implies f(v 1 )  f(v 2 ). Two points do not get mapped together 
under a one-to-one map. It follows that if f is a one-to-one map, then its inverse map f 1 is a function. The

76 2.5 N O N L I N E A R M A P S A N D T H E J AC O B I A N M AT R I X

I NVERSE M APS

A function is a uniquely-defined assignment of a range point for each domain point. (If the domain and range are the same set, we 
call the function a map.) Several domain points may map to the same range point. For f 1 (x, y)  (x 2 , y 2 ), the points (2, 2), (2, 
2), (2, 2) and (2, 2) all map to (4, 4). On the other hand, for f 2 (x, y)  (x 3 , y 3 ), this never happens. A point (a, b) is the 
image of (a 1 ̸ 3 , b 1 ̸ 3 ) only. Thus f 2 is a one-to-one map, and f 1 is not.

An inverse map f 1 automatically exists for any one-to-one map f. The domain of f 1 is the image of f. For the example f 2 (x, y) 

(x 3 , y 3 ), the inverse is f 2 1 (x, y)  (x 1 ̸ 3 , y 1 ̸ 3 ).

To compute an inverse map, set v 1  f(v) and solve for v in terms of v 1 . We demonstrate using f(x, y)  (x  2y, x 3 ). Set

x 1  x  2y

y1 x3 

and solve for x and y. The result is

xy1 1 ̸ 3 

y  (x 1  y 1 1 ̸ 3 ) ̸ 2,

so that the inverse map is f 1 (x, y)  (y 1 ̸ 3 , (x 1  y 1 1 ̸ 3 ) ̸ 2).

inverse map is characterized by the fact that f(v)  w if and only if v  f 1 (w). Because one-to-one implies the existence of an 
inverse, a one-to-one map is also called an invertible map.

✎

E XERCISE T2.8

Show that the H´enon map (2.27) with b  0 is invertible by finding a formula for the inverse. Is the map one-to-one if b  0?

77 T WO -D I M E N S I O N A L M A P S

2.6 S TABLE AND U NSTABLE M ANIFOLDS

A saddle fixed point is unstable, meaning that most initial values near it will move away under iteration of the map. However, unlike 
the case of a source, not all nearby initial values will move away. The set of initial values that converge to the saddle will be 
called the stable manifold of the saddle. We start by looking at a simple linear example.

E XAM PLE 2.17

For the linear map f(x, y)  (2x, y ̸ 2), the origin is a saddle fixed point. The dynamics of this map were shown in Figure 2.14. It is 
clear from that figure that points along the y-axis converge to the saddle 0; all other points diverge to infinity. Unless the 
initial value has x-coordinate 0, the x-coordinate will grow (by a factor of 2 per iterate) and get arbitrarily large.

A convenient way to view the direction of the stable manifold in this case is in terms of eigenvectors. The linear map

f(v)  Av 

2 0 (

0 x 0.5 y )(

)

1 0 , corresponding to the (stretching) eigenvalue 2, and , 0 1 ( ) ( ) corresponding to the (shrinking) eigenvalue 1 ̸ 2. The latter 
direction, the y-axis, is the “incoming” direction, and is the stable manifold of 0. We will call the x-axis the “outgoing” 
direction, the unstable manifold of 0. Another way to describe the unstable manifold in this example is as the stable manifold under 
the inverse

has eigenvector

of the map f 1 (x, y)  ((1 ̸ 2)x, 2y).

Definition 2.18 Let f be a smooth one-to-one map on  2 , and let p be a saddle fixed point or periodic saddle point for f. The stable 
manifold of p, denoted S (p), is the set of points v such that |f n (v)  f n (p)| → 0 as n → . The unstable manifold of p, denoted 
U(p), is the set of points v such that

|f n (v)  f n (p)| → 0 as n → .

E XAM PLE 2.19

The linear map f(x, y)  (2x  5 2 y, 5x  11 2 y) has a saddle fixed point at 0 with eigenvalues 0.5 and 3. The corresponding 
eigenvectors are (1, 1) and

78 2.6 S TA B L E A N D U N S TA B L E M A N I F O L D S

W HAT IS A M ANIFOLD ?

An n-dimensional manifold is a set that locally resembles Euclidean space  n . By “resembles” we could mean a variety of things, and 
in fact, various definitions of manifold have been proposed. For our present purposes, we will mean resemblance in a topological 
sense. A small piece of a manifold should look like a small piece of  n .

A 1-dimensional manifold is locally a curve. Every short piece of a curve can be formed by stretching and bending a piece of a line. 
The letters D and O are 1-manifolds. The letters A and X are not, since each contains a point for which no small neighborhood looks 
like a line segment. These bad points occur at the meeting points of separate segments, like the center of the letter X.

Notable 2-manifolds are the surface of oranges and doughnuts. The space-time continuum of the universe is often described as a 
4manifold, whose curvature due to relativity is an active topic among cosmologists.

In the strict definition of manifold, each point of a manifold must have a neighborhood around itself that looks like  n . Thus the 
letters L and U fail to be 1-manifolds because of their endpoints—small neighborhoods of them look like a piece of half-line (say the 
set of nonnegative real numbers), not a line, since there is nothing on one side. This type of set is called a manifold with 
boundary, although technically it is not a manifold. A M¨obius band is a 2-manifold with boundary because the edge looks locally like 
a piece of half-plane, not a plane. Whole oranges and doughnuts are 3-manifolds with boundary.

One of the goals of Chapter 10 is to explain why a stable or unstable manifold is a topological manifold. Stable and unstable 
manifolds emanate from two opposite sides of a fixed point or periodic orbit. At a saddle point in the plane, they together make an 
“X” through the fixed point, although individually they are manifolds.

79 T WO -D I M E N S I O N A L M A P S

(1, 2), respectively. [According to Appendix A, there is a linear change of coordinates giving the map h(u 1 , u 2 )  (0.5u 1 , 3u 2 
).] Points lying on the line y  x undergo the dynamics v → 0.5v on each iteration of the map. This line is the stable manifold of 0 
for f. Points lying on the line y  2x (the line in the direction of eigenvector (1, 2)) undergo v → 3v under f: this is the unstable 
manifold. These sets are illustrated in Figure 2.18.

E XAM PLE 2.20

Let f(x, y)  (2x  5y, 0.5y). The eigenvalues of f are 2 and 0.5, with corresponding eigenvectors (1, 0) and (2, 1). Points on the 
line in the direction of the vector (2, 1) undergo v → 0.5v on each iteration of f. As a result, successive images flip from one 
side of the origin to the other along the line. This flipping behavior of orbits about the fixed point is shown in Figure 2.19. It is 
characteristic of all fixed points for which the Jacobian has negative eigenvalues, even when the map is nonlinear. A saddle with at 
least one negative eigenvalue is sometimes called a flip saddle. Otherwise it is a regular saddle.

E XAM PLE 2.21

The invertible nonlinear map f(x, y)  (x ̸ 2, 2y  7x 2 ) has a fixed point at 0  (0, 0). To analyze the stability of this fixed point 
we evaluate

y

x

Figure 2.18 Stable and unstable manifolds for regular saddle.

The stable manifold is the solid inward-directed line; the unstable manifold is the solid outward-directed line. Every initial 
condition leads to an orbit diverging to infinity except for the stable manifold of the origin.

80 2.6 S TA B L E A N D U N S TA B L E M A N I F O L D S

y

x

Figure 2.19 Stable and unstable manifolds for flip saddle.

Flipping occurs along the stable manifold (inward-directed line). The unstable manifold is the x-axis.

Df(0, 0) 

0.5 0 0 2 (

)

.

The origin is a saddle fixed point. The eigenvectors lie on the two coordinate axes. What is the relation of these eigenvector 
directions to the stable and unstable manifolds?

In the case of linear maps, the stable and unstable manifolds coincide with the eigenvector directions. For a saddle of a general 
nonlinear map, the stable manifold is tangent to the shrinking eigenvector direction, and the unstable manifold is tangent to the 
stretching eigenvector direction. Since f(0, y)  (0, 2y), the y-axis can be seen to be part of the unstable manifold. For any point 
not on the y-axis, the absolute value of the x-coordinate is nonzero and increases under iteration by f 1 ; in particular, it doesn’t 
converge to the origin. Thus the y-axis is the entire unstable manifold, as shown in Figure 2.20. The stable manifold of 0, however, 
is described by the parabola y  4x 2 ; i.e., S (0)   (x, 4x 2 ) : x    .

✎

E XERCISE T2.9

Consider the saddle fixed point 0 of the map f(x, y)  (x ̸ 2, 2y  7x 2 ) from Example 2.21.

(a) Find the inverse map f 1 .

(b) Show that the set S   (x, 4x 2 ) : x    is invariant under f, that is, if v is in S, then f(v) and f 1 (v) are in S .

81 T WO -D I M E N S I O N A L M A P S

y

x

Figure 2.20 Stable and unstable manifolds for the nonlinear map of Example 2.21.

The stable manifold is a parabola tangent to the x-axis at 0; the unstable manifold is the y-axis.

(c) Show that each point in S converges to 0 under f.

(d) Show that no points outside of S converge to 0 under f.

When a map is linear, the stable and unstable manifolds of a saddle are always linear subspaces. In the case of linear maps on  2 , 
for example, they are lines. For nonlinear maps, as we saw in Example 2.21, they can be curves. The nonlinear examples we have looked 
at so far are not typical; usually, formulas for the stable and unstable manifolds cannot be found directly. Then we must rely on 
computational techniques to approximate their locations. (We describe one such technique in Chapter 10.) One thing you may have 
noticed about the stable manifolds in these examples is that they are always one-dimensional: lines or curves. Just as in the linear 
case, the stable and unstable manifolds of saddles in the plane are always one-dimensional sets. This fact is not immediately 
obviousit is proved as part of the Stable Manifold Theorem in Chapter 10. We will also see that stable and unstable manifolds of 
saddles have a tremendous influence on the underlying dynamics of a system. In particular, their relative positions can determine 
whether or not chaos occurs.

We will leave the investigation of the mysteries of stable and unstable manifolds to Chapter 10. Here we give a small demonstration 
of the subtlety and importance of these manifolds. Drawing stable and unstable manifolds of the H´enon map can illuminate Figure 2.3, 
which showed the basin of the period-two

82 2.6 S TA B L E A N D U N S TA B L E M A N I F O L D S

2.5

2.5 2.5

2.5 2.5

2.5

(a)

(b)

Figure 2.21 Stable and unstable manifolds for a saddle point.

The stable manifolds (mainly vertical) and unstable manifolds (more horizontal) are shown for the saddle fixed point (marked with a 
cross in the lower left corner) of the H´enon map with b  0.3. Note the similarity of the unstable manifold with earlier figures 
showing the H´enon attractor. (a) For a  1.28, the leftward piece of the unstable manifold moves off to infinity, and the rightward 
piece initially curves toward the sink, but oscillates around it in an erratic way. The rightward piece is contained in the region 
bounded by the two components of the stable manifold. (b) For a  1.4, the manifolds have crossed one another.

sink under two different parameter settings. In Figure 2.21, portions of the stable and unstable manifolds of a saddle point near (2, 
2) are drawn. In each case, the upward and downward piece of the stable manifold, which is predominantly vertical, forms the boundary 
of the basin of the period-two sink. (Compare with Figure 2.3.) For a larger value of the parameter a, as in Figure 2.21(b), the 
stable and unstable manifolds intersect, and the basin boundary changes from simple to complicated.

E XAM PLE 2.22

Figure 2.22 shows the relation of the stable and unstable manifolds to the basin of the two-piece attractor for the H´enon map with a  
2, b  0.3. This basin was shown earlier in Figure 2.11. The stable manifold of the saddle fixed point in the lower left corner forms 
the boundary of the attractor basin; the attractor lies along the unstable manifold of the saddle.

83 T WO -D I M E N S I O N A L M A P S

2.5

2.5 2.5

2.5

Figure 2.22 A two-piece attractor of the H´enon map.

The crosses mark 100 points of a trajectory lying on a two-piece attractor. The basin of attraction of this attractor is white; the 
shaded points are initial conditions whose orbits diverge to . The saddle fixed point circled at the lower left is closely related to 
the dynamics of the attractor. The stable manifold of the saddle, shown in black, forms the boundary of the basin of the attractor. 
The attractor lies along the unstable manifold of the saddle, which is also in black.

E XAM PLE 2.23

Figure 2.23(a) shows 18 fixed points (large crosses) and 38 period-two orbits (small crosses) for the time-2  map of the forced 
damped pendulum (2.10) with c  0.05, 
  2.5. The orbits were found by computer approximation methods; there may be more. None of these orbits are sinks; they coexist with 
a complicated attracting orbit shown in Figure 2.7. Exactly half of the 56 orbits shown are flip

84 2.6 S TA B L E A N D U N S TA B L E M A N I F O L D S

4.5

3.5  



 



(a)

(b)

Figure 2.23 The forced damped pendulum.

(a) Periodic orbits for the time-2  map of the pendulum with parameters c  0.05, 
  2.5. The large crosses denote 18 fixed points, and the small crosses, 38 period-two orbits. (b) The stable and unstable manifolds 
of the largest cross in (a). The unstable manifold is drawn in black; compare to Figure 2.7. The stable manifold is drawn in gray 
dashed curves. The manifolds overlay the periodic orbits from (a)note that without exception these orbits lie close to the unstable 
manifold.

saddles; the rest are regular saddles. The largest cross in Figure 2.23(a) is singled out, and its stable and unstable manifolds are 
drawn in Figure 2.23(b).

Exercise 10.6 of Chapter 10 states that a stable manifold cannot cross itself, nor can it cross the stable manifold of another fixed 
point. However, there is no such restriction for a stable manifold crossing an unstable manifold.

The discovery that stable and unstable manifolds of a fixed point can intersect was made by Poincar´e. He made this observation in 
the process of fixing his entry to King Oscar’s contest. (In his original entry he made the assumption that they could not cross.) 
Realizing this possibility was a watershed in the knowledge of dynamical systems, whose implications are still being worked out 
today.

Poincar´e was surprised to see the extreme complexity that such an intersection causes in the dynamics of the map. If p is a fixed or 
periodic point, and if h 0  p is a point of intersection of the stable and unstable manifold of p, then h 0 is called a homoclinic 
point. For starters, an intersection of the stable and unstable manifolds of a single fixed point (called a homoclinic intersection) 
immediately

85 T WO -D I M E N S I O N A L M A P S

h 1

h 2

h 3

h 0

h -1

p

h -3

h -2

Figure 2.24 A schematic view of a homoclinic point h 0 .

The stable manifold (solid curve) and unstable manifold (dashed curve) of the saddle fixed point p intersect at h 0 , and therefore 
also at infinitely many other points. This figure only hints at the complexity. Poincar´e showed that if a circle was drawn around 
any homoclinic point, there would be infinitely many homoclinic points inside the circle, no matter how small its radius.

forces infinitely many such intersections. Poincar´e drew diagrams similar to Figure 2.24, which displays the infinitely many 
intersections that logically follow from the intersection h 0 .

To understand the source of this complexity, first notice that a stable manifold, by definition, is an invariant set under the map f. 
This means that if h0  is a point on the stable manifold of a fixed point p, then so are h 1  f(h 0 ) and h 1  f 1 (h 0 ). This is 
easy to understand: if the orbit of h 0 eventually converges to p under f, then so must the orbits of h 1 and h 1 , being one step 
ahead and behind of h 0 , respectively. In fact, if h 0 is a point on a stable manifold of p, then so is the entire (forward and 
backward) orbit of h 0 . By the same reasoning, the unstable manifold of a fixed point is also invariant.

Once a point like h 0 in Figure 2.24 lies on both the stable and unstable manifolds of a fixed point, then the entire orbit of h 0 
must lie on both manifolds, because both manifolds are invariant. Remember that the stable manifold is directed toward the fixed 
point, and the unstable manifold leads away from it. The result is a configuration drawn schematically in Figure 2.24, and generated 
by a computer for the H´enon map in Figure 2.21(b).

86 2.7 M AT R I X T I M E S C I R C L E E Q UA L S E L L I P S E

The key fact about a homoclinic intersection point is that it essentially spreads the sensitive dependence on initial 
conditions—ordinarily situated at a single saddle fixed point—throughout a widespread portion of state space. Figures 2.21 and 2.24 
give some insight into this process. In Chapter 10, we will return to study this mechanism for manufacturing chaos.

2.7 M ATRIX T IMES C IRCLE E QUALS E LLIPSE

Near a fixed point v 0 , we have seen that the dynamics essentially reduce to a single linear map A  Df(v 0 ). If a map is linear 
then its action on a small disk neighborhood of the origin is just a scaled–down version of the effect on the unit disk. We found 
that the magnitudes of the eigenvalues of A were decisive for classifying the fixed point. The same is true for a period-k orbit; in 
that case the appropriate matrix A is a product of k matrices.

In case the orbit is not periodic (which is one of our motivating situations), there is no magic matrix A. The local dynamics in the 
vicinity of the orbit is ruled, even in its linear approximation, by an infinite product of usually nonrepeating Df(v 0 ). The role 
of the eigenvalues of A is taken over by Lyapunov numbers, which measure contraction and expansion. When we develop Lyapunov numbers 
for many-dimensional maps (Chapter 5), it is this infinite product that we will have to measure or approximate in some way.

To visualize what is going on in cases like this, it helps to have a way to calculate the image of a disk from the matrix 
representing a linear map. For simplicity, we will choose the disk of radius one centered at the origin, and a square matrix. The 
image will be an ellipse, and matrix algebra explains how to find that ellipse.

The technique (again) involves eigenvalues. The image of the unit disk N under the linear map A will be determined by the 
eigenvectors and eigenvalues of AA T , where A T denotes the transpose matrix of A (formed by exchanging the rows and columns of A). 
The eigenvalues of AA T are nonnegative for any A. This fact can be found in Appendix A, along with the next theorem, which shows how 
to find the explicit ellipse AN.

Theorem 2.24 Let N be the unit disk in  m , and let A be an m m matrix. Let s 1 2 , . . . , s m 2 and u 1 , . . . , u m be the 
eigenvalues and unit eigenvectors, respectively, of the m m matrix AA T . Then

1. u 1 , . . . , u m are mutually orthogonal unit vectors; and

2. the axes of the ellipse AN are s i u i for 1 i m.

87 T WO -D I M E N S I O N A L M A P S

Check that in Example 2.5, the map A gives s 1  a, s 2  b, while u 1 and u 2 are the x and y unit vectors, repectively. Therefore a 
and b are the lengths of the axes of the ellipse AN. For the nth iterate of A, represented by the matrix A n , we find ellipse axes 
of length a n and b n for A n N, the nth image of the unit disk.

In Example 2.5, the axes of the ellipse AN are easy to find. Each axis is an eigenvector not only of AA T but also of A, whose length 
is the corresponding eigenvalue of A. In general (for nonsymmetric matrices), the eigenvectors of A do not give the directions along 
which the ellipse lies, and it is necessary to use Theorem 2.24. To see how Theorem 2.24 applies in general, we’ll return for a look 
at our three important examples.

E XAM PLE 2.25

[Distinct real eigenvalues.] Let

A(x)  Ax 

.8 0 (

.5 x 1.3 y )( )

.

(2.36)

The eigenvalues of the matrix A are 0.8 and 1.3, with corresponding eigenvectors 1 1 and , respectively. From this it is clear that 
the fixed point at the origin 0 1 ( ) ( ) is a saddle—the two eigenvectors give directions along which the fixed point attracts and 
repels, respectively. The attracting direction is illustrated by

1 1 (0.8)n  A n  (0.8) n  , 0 0 0 ( ) ( ) ( )

(2.37)

and the repelling direction by

1 1 (1.3)n  A n  (1.3) n  . 1 1 (1.3)n  ( ) ( ) ( )

(2.38)

The stable manifold of the origin saddle point is y  0, and the unstable manifold is y  x. Points along the x-axis move directly 
toward the origin under iteration by A, and points along the line y  x move toward infinity. Since we know the nth iterate of the 
unit circle is an ellipse with one growing direction and one shrinking direction, we know that in the limit the ellipses become long 
and thin. The ellipses A n N representing higher iterates of the unit disk gradually line up 1 along the dominant eigenvector of A.

1 ( ) The first few images of the unit disk under the map A can be found using Theorem 2.24, and are graphed in Figure 2.25. For an 
application of Theorem

88 2.7 M AT R I X T I M E S C I R C L E E Q UA L S E L L I P S E

y=x

5

A 6 N

4

3

2

A 4 N

N

-5 -4

-3 -2

2

3

4

5

-2

-3

-4

-5

Figure 2.25 Successive images of the unit circle N for a saddle fixed point. The image of a circle under a linear map is an ellipse. 
Successive images are therefore also ellipses, which in this example line up along the expanding eigenspace.

2.24, we calculate the first iterate of the unit disk under A. Since

.8 .5 .8 0 .89 .65 AA T   , 0 1.3 .5 1.3 .65 1.69 ( )( ) ( )

(2.39)

.873 .488 and , with .488 .873 ( ) ( ) eigenvalues .527 and 2.053, respectively. Taking square roots, we see that the ellipse AN has 
principal axes of lengths  √ .527  .726 and  √ 2.053  1.433. The ellipse AN, along with A 4 N and A 6 N, is illustrated in Figure 
2.25.

the unit eigenvectors of AA T are (approximately)

E XAM PLE 2.26

[Repeated eigenvalue.] Even in the sink case, the ellipse A n N can grow a little in some direction before shrinking for large n. 
Consider the example

2 1 A 3 2 . 0 ( 3 )

(2.40)

✎

E XERCISE T2.10

Use Theorem 2.24 to calculate the axes of the ellipse AN from (2.40). Then verify that the ellipses A n N shrink to the origin as n 
→ .

89 T WO -D I M E N S I O N A L M A P S

N

AN

A 2 N A 4 N A 6 N

Figure 2.26 Successive images of the unit circle N under a linear map A in the case of repeated eigenvalues.

The nth iterate A n N of the circle lies wholly inside the circle for n 
 6. In this case, the origin is a sink.

The first few iterates of the unit circle N are graphed in Figure 2.26. The ellipse AN sticks out of the unit disk N. Further 
iteration by A continues to roll the ellipse to lie parallel to the x-axis and to eventually shrink it to the origin, as the 
calculation of Exercise T2.10 requires.

E XAM PLE 2.27

[Complex eigenvalues.] Let

A

a b (

b a

)

.

(2.41)

The eigenvalues of this matrix are a  bi. Calculating AA T yields

a 2  b 2 0 AA T  , 0 a 2  b2  ( )

(2.42)

so it follows that the image of the unit disk N by A is again a disk of radius √ a 2  b 2 . The matrix A rotates the disk by arctan 
b ̸ a and stretches by a factor of √ a 2  b 2 on each iteration. The stability result is the same as in the previous two cases: if 
the absolute value of the eigenvalues is less than 1, the origin is a sink; if greater than 1, a source.

90 2.7 M AT R I X T I M E S C I R C L E E Q UA L S E L L I P S E

2

-2

2

N

A 3 N

A 2 N

AN

-2

Figure 2.27 Successive images of the unit circle N . The origin is a source with complex eigenvalues.

In Figure 2.27, the first few iterates A n N of the unit disk are graphed for a  1.2, b  0.2. Since a 2  b 2  1, the origin is a 
source. The radii of the images of the disk grow at the rate of √ 1.2 2  .2 2  1.22 per iteration, and the disks turn 
counterclockwise at the rate of arctan(.2 ̸ 1.2)  9.5 degrees per iteration.

91 T WO -D I M E N S I O N A L M A P S

☞ CHALLENGE 2

Counting the Periodic Orbits of Linear Maps on a Torus

IN CHAPTER 1, we investigated the properties of the linear map f(x)  3x (mod 1). This map is discontinuous on the interval [0, 1], 
but continuous when viewed as a map on a circle. We found that the map had infinitely many periodic points, and we discussed ways to 
count these orbits.

We will study a two-dimensional map with some of the same properties in Challenge 2. Consider the map S defined by a 2 2 matrix

A

a c (

b d

)

with integer entries a, b, c, d, where we define S(v)  Av (mod 1). The domain for the map will be the unit square [0, 1] [0, 1]. Even 
if Av lies outside the unit square, S(v) lies inside if we count modulo one. In general, S will fail to be continuous, in the same 
way as the map in Example 1.9 of Chapter 1.

For example, assume

A

2 1 (

1

1

)

.

(2.43)

Consider the image of the point v  (x, 1 ̸ 2) under S. For x slightly less than 1 ̸ 2, the image S(v) lies just below (1 ̸ 2, 1). For x 
slightly larger than 1 ̸ 2, the image S(v) lies just above (1 ̸ 2, 0), quite far away. Therefore S is discontinuous

at (1 ̸ 2, 1 ̸ 2).

We solved this problem for the 3x mod 1 map in Chapter 1 by sewing together the ends of the unit interval to make a circle. Is there 
a geometric object for which S can be made continuous? The problem is that when the image value 1 is reached (for either coordinate x 
or y), the map wants to restart at the image value 0.

The torus  2 is constructed by identifying the two pairs of opposite sides of the unit square in  2 . This results in a 
two-dimensional object resembling a doughnut, shown in Figure 2.28. We have simultaneously glued together the x-axis at 0 and 1, and 
the y-axis at 0 and 1. The torus is the natural domain for maps that are formed by integer matrices modulo one.

92 CHALLENGE 2

(a)

(b)

(c)

Figure 2.28 Construction of a torus in two easy steps.

(a) Begin with unit square. (b) Identify (glue together) vertical sides. (c) Identify horizontal sides.

Given a 2 2 matrix A, we can define a map from the torus to itself by multiplying the matrix A times a vector (x, y), followed by 
taking the output vector modulo 1. If the matrix A has integer entries, then the map so defined is continuous on the torus. In the 
following steps we derive some fundamental properties of torus maps, and then specialize to the particular map (2.43), called the cat 
map.

Assume in Steps 1–6 that A has integer entries, and that the determinant of A, det(A)  ad  bc, is nonzero.

Step 1 Show that the fact that A is a 2 2 matrix with integer entries implies that the torus map S(v)  Av (mod 1) is a continuous map 
on the torus  2 . (You will need to explain why the points (0, y) and (1, y), which are identified together on the torus, map to the 
same point on the torus. Similarly for (x, 0) and (x, 1).)

) n1 ,n2 .

Step 2

(a) Show that A

x  n1  y  n2  (

A

x y ( )

(mod 1) for any integers

(b) Show that S 2 (v)  A 2 v (mod 1).

(c) Show that S n (v)  A n v (mod 1) for any positive integer n.

Step 2 says that in computing the nth iterate of S, you can wait until the end to apply the modulo 1 operation.

A real number r is rational if r  p ̸ q, where p and q are integers. A number that is not rational is called irrational. Note that the 
sum or product of two rational numbers is rational, the sum of a rational and an irrational is irrational,

93 T WO -D I M E N S I O N A L M A P S

and the product of a rational and an irrational is irrational unless the rational is zero.

Step 3 Assume that A has no eigenvalue equal to 1. Show that S(v)  v implies that both components of v are rational numbers.

Step 4 Assume that A has no eigenvalues of magnitude one. Since the eigenvalues of A n are the nth powers of the eigenvalues of A 
(see Appendix A), this assumption guarantees that for all n, the matrix A n does not have an eigenvalue equal to 1. Show that a point 
v  (x, y) in the unit square is eventually periodic if and only if its coordinates are rational. (The 3x (mod 1) map of Chapter 1 had 
a similar property.) [Hint: Use Step 3 to show that if any component is irrational, then v cannot be eventually periodic. If both 
components of v are rational, show that there are only a finite number of possibilities for iterates of v.]

Step 5 Show that the image of the map S covers the square precisely |det(A)| times. More precisely, if v 0 is a point in the square, 
show that the number of solutions of S(v)  v 0 is |det A|. [Hint: Draw the image under A of the unit square in the plane. It is a 
parallelogram with one vertex at the origin and three other vertices with integer coordinates. The two sides with the origin as 
vertex are the vectors (a, c) and (b, d). The area of the parallelogram is therefore |det(A)|. Show that the number of solutions of 
Av  v 0 (mod 1) is the same for all v 0 in the square. Therefore the parallelogram can be cut up by mod 1 slices and placed onto the 
square, covering it |det(A)| times. See Figure 2.29.]

(a)

(b)

Figure 2.29 Dynamics of the cat map.

(a) The unit square and its image under the cat map (2.43). (b) Since det(A)  1, the image of the unit square, modulo 1, covers the 
unit square exactly once.

94 CHALLENGE 2

Step 6 Define the trace of A by tr(A)  a  d, the sum of the diagonal entries. Prove that the number of fixed points of S on the torus 
is |det A  tr A  1|, as long as that number is not zero. [Hint: Apply Step 5 to the matrix A  I.]

Now we study a particular choice of A. Define the matrix 2 1 A 1 1 ( ) as in Equation (2.43). The resulting map S, called the “cat 
map”, is shown in Figure 2.30.

Step 7 Let F n denote the nth Fibonacci number, where F 0  F 1  1, and where F n  F n1  F n2 for n 
 2. Prove that

F 2n F2n1  An  . F 2n1 F2n2  ( )

Step 8 Find all fixed points and period-two orbits of the cat map. [Hint: For the period-2 points, find all solutions of a c a   5 3  
m b d b a c c   3 (2.44) 2  n b d d

(a)

(b)

Figure 2.30 An illustration of multiplication by the matrix A in (2.43). (a) Cat in unit square. (b) Image of the unit square under 
the matrix A.

95 T WO -D I M E N S I O N A L M A P S

where a, b, c, d, m, n are integers.] Answers: (0,0) is the only fixed point; the period-2 orbits are:

.2 .4 {(

.8 .6 ) ( )}

,

and

.4 .8 {(

.6 .2 ) (

,

)}

.

Step 9 Use Step 6 to find a formula for the number of fixed points of S n . (Answer: |(F 2n  1)(F 2n2  1)  F 2n1 2 |.)

The formula in Step 9 can be checked against Figure 2.31 for low values of n. For example, Figure 2.31(a) shows a plot of the fixed 
points of S 4 in the unit square. There are 45 points, each with coordinates expressible as (i ̸ 15, j ̸ 15) for some integers i, j. 
One of them is a period-one point of S (the origin), and four of them are the period-two points (two different orbits) of S found in 
Step 8. That leaves a total of 10 different period-four orbits. In counting, remember that the points are defined modulo 1, so that a 
point on the boundary of the square will also appear as the same point on the opposite boundary. The period-five points in Figure 
2.31(b) are somewhat easier to count; they are the 120 points of form (i ̸ 11, j ̸ 11) where 0 i, j 10, omitting the origin, which is a 
period-one point. There are 24 period-five orbits. The period-three points show up as the medium-sized crosses in Figure 2.31(c). 
They are the 15 points of form (i ̸ 4, j ̸ 4) where 0 i, j 3, again omitting the origin. Can you guess the denominators of the 
period-six points? See Step 12 for the answer.

(a)

(b)

(c)

Figure 2.31 Periodic points of the cat map on the torus.

(a) Period-four points (small crosses), period-two points (large crosses). (b) Periodfive points. (c) Period-six points (small 
crosses), period-three points (medium crosses), period-two points (large crosses)

96 CHALLENGE 2

Step 10 Prove the identities F n F n2  F n1 2  (1) n and F 2n  F n 2  F n1 2 for Fibonacci numbers, and use them to simplify the 
formula for the number of fixed points of S n to (F n  F n2 ) 2  2  2(1) n .

Step 11 Write out the periodic table for S (list number of periodic points for each period) for periods up to 10. See Table 1.3 of 
Chapter 1 for the form of a periodic table. Compare this with Figure 2.31 where applicable. Show that S has periodic points of all 
periods.

Step 12 Here is the formula for the denominator of the period n orbits. The orbits consist of points a ̸ b where

F n  F n2 if n is odd and n 5F n1 if n is even { For example, the denominator of the period-two points is 5F 1  5, and the 
denominator of the period-three points is F 3  F 1  4. Confirm this with the answers from Step 8 and Figure 2.31(a). Prove this 
formula for general n. [Thanks to Joe Miller for simplifying this formula.]

 1

b

Postscript. How robust are the answers you derived in Challenge 2? If you’re still reading, you have done an exhaustive accounting of 
the periodic orbits for the cat map (2.43). Does this accounting apply only to the cat map, or are the same formulas valid for 
cat-like maps?

Using a few advanced ideas, you can show that the results apply as well to maps in a neighborhood of the cat map in “function space”, 
including nearby nonlinear maps. Exactly the same number of orbits of each period exist for these maps. The reasoning is as follows. 
As we will see in Chapter 11, as a map is perturbed (say, by moving a parameter a), a fixed point p of f a k cannot appear or 
disappear (as a moves) without an eigenvalue of the matrix Df a k (p) crossing through 1. (At such a crossing, the Jacobian of f a k  
I is singular and the implicit function theorem is violated, allowing a solution of f k  I  0 to appear or disappear as the map is 
perturbed.)

The eigenvalues of the cat map A are e 1  (3  √ 5) ̸ 2  2.618 and e 2  (3  √ 5) ̸ 2  0.382, and the eigenvalues of A k are e 1 k and 
e 2 k . Since eigenvalues move continuously as a parameter in the matrix is varied, any map whose Jacobian entries are close to those 
of the cat map will have eigenvalues close to |e 1 | and |e 2 |, bounded away from 1, and so the eigenvalues of Df k (p) will not 
equal 1. This rules out the appearance and disappearance of periodic points for linear and nonlinear maps sufficiently similar to the 
cat map, showing that the same formulas derived in Challenge 2 hold for them as well. Nonlinear maps of the torus close to the cat 
map such as

f(x, y)  (2x  y  a cos 2  x, x  y  b sin 2  y) (mod 1)

where |a|, |b| are sufficiently small have this property, and so have the same number of periodic points of each period as the cat 
map.

97 T WO -D I M E N S I O N A L M A P S

E XERCISES

2.1. For each of the following linear maps, decide whether the origin is a sink, source, or saddle.

(a)

4 ( 1

1 (b) ( 1

1 ̸ )

̸ 4

2 3

̸

4

30

3

)

(c)

0.4 ( 0.4

)

2.4

1.6

n 4.5 8 6 2.2. Find lim . n→  2 3.5 9 ( ) ( )

2.3. Let g(x, y)  (x 2  5x sources, or saddles.

 y, x2 

). Find and classify the fixed points of g as sinks,

2.4. Find and classify all fixed points and period-two orbits of the H´enon map (2.27) with

(a) a  0.56 and b  0.5

(b) a  0.21 and b  0.6

2.5. Let f(x, y, z)  (x 2 y, y 2 , xz  y) be a map on  3 . Find and classify the fixed points of f.

2.6. Let f(x, y)  (sin 3  x, 2 y ). Find all fixed points and their stability. Where does the orbit of each initial value go?

2.7. Set b  0.3 in the H´enon map (2.27). (a) Find the range of parameters a for which the map has one fixed sink and one saddle 
fixed point. (b) Find the range of parameters a for which the map has a period-two sink.

2.8. Calculate the image ellipse of the unit disk under each of the following maps.

(a)

2 ( 2

)

0.5 0.5

(b)

2 ( 2

)

1

2

What are the areas of these ellipses?

2.9. Find the inverse map for the cat map defined in (2.43). Check your answer by composing with the cat map.

2.10. (a) Find a 2 2 matrix A  I with integer entries, a rational number x, and an irrational number y such that S(x, y)  (x, y). 
Here S is the mod 1 map associated to A. (b) Same question but require x and y to be irrational. According to Step 4 of Challenge 2, 
each of your answers must have an eigenvalue equal to 1.

2.11. Let a be a vector in R m and M be an m m matrix. Define f(x)  Mx  a. Find a condition on M (specifically, on the eigenvalues of 
M) that guarantees that f has exactly one fixed point.

98 LAB VISIT 2

☞ LAB VISIT 2

Is the Solar System Stable?

KING OSCAR’S CONTEST in 1889 was designed to answer the question of whether the solar system is stable, once and for all. The actual 
result clarified just how difficult the question is. If the contest were repeated today, there would be no greater hope of producing 
a definitive answer, despite (or one might say, because of) all that has been learned about the problem in the intervening century.

Poincar´e’s entry showed that in the presence of homoclinic intersections, there is sensitive dependence on initial conditions. If 
this exists in our solar system, then long-term predictability is severely compromised. The positions, velocities, and masses of the 
planets of the solar system are known with considerably more precision than was known in King Oscar’s time. However, even these 
current measurements fall far short of the accuracy needed to make long-term predictions in the presence of sensitive dependence on 
initial conditions. Two key questions are: (1) whether chaos exists in planetary trajectories, and (2) if there are chaotic 
trajectories, whether the chaos is sufficiently pronounced to cause ejection of a planet from the system, or a planetary collision.

The question of whether chaos exists in the solar system has led to innovations in theory, algorithms, computer software, and 
hardware in an attempt to perform accurate long-term solar system simulations. In 1988, Sussman and Wisdom reported on an 845 Myr 
(Myr denotes one million years) integration of the gravitational equations for the solar system. This integration was performed in a 
special-purpose computer that they designed for this problem, called the Digital Orrery, which has since been retired to the 
Smithsonian Institution in

Sussman, G. J., Wisdom, J., “Numerical evidence that the motion of Pluto is chaotic.” Science 241, 433-7 (1988).

Laskar, J., “A numerical experiment on the chaotic behaviour of the solar system.” Nature 338, 237-8 (1989).

Sussman, G.J., Wisdom, J., “Chaotic evolution of the solar system.” Science 257, 56-62 (1992).

Laskar, J., Robutel, P., “The chaotic obliquity of the planets.” Nature 361, 608-612 (1993).

Touma, J., Wisdom, J., “The chaotic obliquity of Mars.” Science 259, 1294-1297 (1993).

99 T WO -D I M E N S I O N A L M A P S

Figure 2.32 Comparison of two computer simulations.

The difference in the position of Pluto in two simulations with slightly different initial conditions is plotted as a function of 
time. The vertical scale is ln of the difference, in units of AU (astronomical units).

Washington, D.C. A surprising result of the integration was an indication of sensitive dependence in the orbit of Pluto.

Figure 2.32 shows exponential divergence of nearby trajectories over a 100 Myr simulation, done by Sussman and Wisdom in 1992 with a 
successor of the Digital Orrery, built as a collaboration between MIT and Hewlett-Packard. They made two separate computer runs of 
the simulated solar system. The runs were identical except for a slight difference in the initial condition for Pluto. The curve 
shows the distance between the positions of Pluto for the two solar system simulations.

The plot in Figure 2.32 is semilog, meaning that the vertical axis is logarithmic. Assume that the distance d between the Plutos in 
the two almost parallel universes grows exponentially with time, say as d  ae kt . Then log d  log a  kt, which is a linear relation 
between log d and t. If we plot log d versus t, we will get a line with slope k. This is what Figure 2.32 shows in rough terms. In 
Chapter 3 the slope k will be called the Lyapunov exponent. A careful look at the figure yields the slope to be approximately 1 ̸ 12, 
meaning that the distance between

100 LAB VISIT 2

nearby initial conditions is multiplied by a factor of e 1 ̸ 12 each million years, or by a factor of e  2.718 each 12 million years. 
We can say that the exponential separation time for Pluto is about 12 million years on the basis of this simulation.

Laskar, working at the time at the Bureau des Longitudes in Paris, used quite different techniques to simulate the solar system 
without Pluto, and concluded that the exponential separation time for this system is on the order of 5 million years. He attributed 
the chaotic behavior to resonances among the inner planets Mercury, Venus, Earth, and Mars. Recent simulations by Sussman and Wisdom 
have also arrived at the approximate value of 5 million years for some of the inner planets.

To put these times in perspective, note that e 19  1.5 10 8 , which is the number of kilometers from the sun to the earth, or one 
astronomical unit. Therefore a difference or uncertainty of 1 km in a measured position could grow to an uncertainty of 1 
astronomical unit in about 19 time units, or 19 5  95 Myrs. The solar system has existed for several billions of years, long enough 
for this to happen many times over.

The finding of chaos in solar system trajectories does not in itself mean that the solar system is on the verge of disintegration, or 
that Earth will soon cross the path of Venus. It does, however, establish a limit on the ability of celestial mechanics to predict 
such an event in the far future.

A more recent conclusion may have an impact on life on Earth within a scant few millions of years. The two research groups mentioned 
above published articles within one week of one another in 1993 regarding the erratic obliquity of planets in the solar system. The 
obliquity of a planet is the tilt of the spin axis with respect to the “plane” of the solar system. The obliquity of Earth is 
presently about 23.3 ◦ , with estimated variations over time of 1 ◦ either way. Large variations in the obliquity, for example 
those that would turn a polar icecap toward the sun for extended periods, would have a significant effect on climate.

The existence of a large moon has a complicating and apparently stabilizing effect on the obliquity of Earth. A more straightforward 
calculation can be made for Mars. Laskar and Robutel found that in a 45 Myr simulation, the obliquity of Mars can oscillate 
erratically by dozens of degrees, for some initial configurations. For other initial conditions, the oscillations are regular.

Figure 2.33(a) shows the variation of the obliquity of Mars for 80 Myrs into the past, from a computer simulation due to Touma and 
Wisdom. This simulation takes the present conditions as initial conditions and moves backwards in time. Figure 2.33(b) is a 
magnification of part (a) for the last 10 Myrs only. It shows an abrupt transition about 4 Myrs ago, from oscillations about 35 ◦ 
obliquity to oscillations about 25 ◦ obliquity.

101 T WO -D I M E N S I O N A L M A P S

Figure 2.33 The obliquity of Mars.

(a) The result of a computer simulation of the solar system shows that the obliquity of Mars undergoes erratic variation as a 
function of time. (b) Detail from (a), showing only the last 10 million years. There is an abrupt transition about 4 million years 
ago.

102 LAB VISIT 2

No one knows whether the graphs shown here are correct over a severalmillion-year time range. Perhaps core samples from the poles of 
Mars (not yet obtained) could shed light on this question. If the initial position, velocity, mass, and other physical parameters of 
Mars used in the simulations were changed from the values used in the simulations, the results would be different, because of the 
sensitive dependence of the problem. Any calculation of the obliquity or the position of Mars for many millions of years can only be 
considered as representative of the wide range of possibilities afforded by chaotic dynamics.

While sensitive dependence on initial conditions causes unpredictability at large time scales, it can provide opportunity at shorter, 
predictable time scales. One of the first widely available English translations of Poincar´e’s writings on celestial mechanics was 
commissioned by the U.S. National Aeronautics and Space Administration (NASA). Several years ago, their scientists exploited the 
sensitive dependence of the three-body problem to achieve a practical goal.

In 1982, NASA found itself unable to afford to send a satellite to the Giacobini-Zinner comet, which was scheduled to visit the 
vicinity of Earth’s orbit in 1985. It would pass quite far from the current Earth position (50 million miles), but near Earth’s 
orbit. This led to the possibility that a satellite already near Earth’s orbit could be sent to the correct position with a 
relatively low expenditure of energy.

A 1000-lb. satellite called ISEE-3 had been launched in 1978 to measure the solar wind and to count cosmic rays. ISEE-3 was parked in 
a “halo orbit”, centered on the Lagrange point L 1 . The halo orbit is shown in Color Plate 13. A Lagrange point is a point of 
balance between the gravitational pull of the Earth and Sun. In the rotating coordinate system in which the Earth and Sun are fixed, 
the L 1 point is an unstable equilibrium. Lagrange points are useful because little energy is required to orbit around them.

The ISEE-3 satellite was nearing the end of its planned mission, and had a limited amount of maneuvering fuel remaining. NASA 
scientists renamed the satellite the International Comet Explorer (ICE) and plotted a three-year trajectory for the Giacobini-Zinner 
comet. The new trajectory took advantage of near collisions with the Moon, or “lunar swingbys”, to make large changes in the 
satellite’s trajectory with small fuel expenditures. The first thruster burn, on June 10, 1982, changed the satellite’s velocity by 
less than 10 miles per hour. In all, 37 burns were needed to make small changes in the trajectory, resulting in 5 lunar swingbys.

Dynamical motion of gravitational bodies is especially sensitive at a swingby. Since the distance between the two bodies is small, 
the forces between them is relatively large. This is where small changes can have a large effect.

103 T WO -D I M E N S I O N A L M A P S

The first lunar swingby occurred on March 30, 1983. A schematic picture of the satellite trajectory is shown in Color Plate 14. The 4 
other near collisions with the moon are shown in the following Color Plates 15–16. During the final swingby on Dec. 22, 1983, denoted 
by 5 in the figure, the satellite passed within 80 miles of the surface of the moon on its way toward the comet. ICE passed through 
the tail of the Giacobini-Zinner comet on Sept. 11, 1985, exactly as planned.

104 CHAPTER THREE

Chaos T

HE CONCEPT of an unstable steady state is familiar in science. It is not possible in practice to balance a ball on the peak of a 
mountain, even though the configuration of the ball perfectly balanced on the peak is a steady state. The problem is that the 
trajectory of any initial position of the ball near, but not exactly at, the steady state, will evolve away from the steady state. We 
investigated sources and saddles, which are unstable fixed points of maps, in Chapters 1 and 2.

What eventually happens to the ball placed near the peak? It moves away from the peak and settles in a valley at a lower altitude. 
The valley represents a stable steady state. One type of behavior for an initial condition that begins near an unstable steady state 
is to move away and be attracted by a stable steady state, or perhaps a stable periodic state.

We have seen this behavior in maps of the real line. Consider an initial condition that is near a source p of a map f. At the 
beginning of such an orbit,

105 C H AO S

unstable behavior is displayed. Exponential separation means that the distance between the orbit point and the source increases at an 
exponential rate. Each iteration multiplies the distance between them by |f′(p)|  1. We s
