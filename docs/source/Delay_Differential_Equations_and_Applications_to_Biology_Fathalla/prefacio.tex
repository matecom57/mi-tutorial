\documentclass[12pt]{article}
\usepackage{lingmacros}
\usepackage{tree-dvips}
\begin{document}


Delay differential equations (DDEs) or functional differential equations arise in models representing biological phenomena when the 
time-delays occurring in these phenomena are considered. Mathematical modeling using such DDEs is widely applied for performing analysis and 
predictions in various areas of life sciences, such as population dynamics, epidemiology, immunology, physiology, and neural networks. The 
memory or time-delays in these models are related to the duration of certain hidden processes, such as the stages of a life cycle, the time 
between the start of a cell infection and the production of new viruses, the infection period, and the immune period. In ordinary 
differential equations (ODEs), the unknown state and its derivatives are evaluated at the same time instant. In DDEs, however, the evolution 
of the system at a certain time instant depends on the past history/memory. Introduction of such time-delays in a differential model 
significantly improves the dynamics of the model and increases the complexity of the system. Therefore, studying qualitative and quantitative 
behaviors of such class of differential equations is essential.

Here, parts of the theory of differential equations and functional differential equations are discussed that can or have been applied to 
modeling biological systems. Models with fractional-order derivatives and models with environmental noise (stochastic models) are also 
investigated. This book is different from other books on this topic; this is because both qualitative and quantitative features of DDEs and 
their applications in biosciences are studied herein. This book covers various important topics related to DDEs, including numerical methods, 
stability, inverse problems, parameter estimations, sensitivity analysis, optimal control, and biological systems with memory (time-delays). 
This book is useful to a wide range of mathematicians and specialists in the fields of mathematical biology, mathematical modeling, life 
sciences, immunology, and infectious diseases. Thus, it can be recommended as a textbook for graduate and postgraduate students, bridging the 
gap between mathematics and various areas of bioscience research.

In this monograph, we discuss a wide range of DDEs with integer- and fractionalorder derivatives and show how they have a richer mathematical 
framework for the analysis of dynamical systems (compared with differential equations without memory). This monograph consists of two parts, 
organized into 13 chapters. Part I (Chaps. 1–7) is devoted to the study of the qualitative and quantitative features of DDEs, whereas Part II 
(Chaps. 8–13) discusses certain applications of DDEs in biosciences. Chapter I provides a brief introduction and discusses the qualitative 
features of DDEs. In Chapter II, we study numerical solutions and methods for DDEs. In Chapter III, we investigate stability concepts of the 
numerical schemes of DDEs. Chapter IV provides unconditionally stable numerical schemes for integro-DDEs, which are suitable for stiff and 
non-stiff problems. In Chapter V, we explore the inverse problem with DDEs as well as parameter estimation and parameter identifiably of 
DDEs. In Chapter VI, we estimate sensitivity functions and analysis of DDEs to evaluate how the state variable can vary with respect to small 
variations in the initial data and parameters (or constant lags) appearing in the model. Chapter VII discusses certain features of stochastic 
delay differential equations (SDDEs), and we also study some efficient numerical schemes for SDDEs. Chapter VIII shows how DDEs have, 
prospectively, more interesting dynamics for epidemics and infectious diseases. In Chapter IX, we study DDEs with tumor-immune interaction in 
presence of external treatment and optimal control. In Chapter X, we investigate DDEs for ecological and predator-prey systems. In Chapter 
XI, we explore fractional-order DDEs for predator-prey systems. In Chapter XII, we study the dynamics of Hepatitis C viral infection through 
fractional-order DDEs. In Chapter XIII, we study stochastic delay differential equations for the spread of COVID-19. In the last chapter, we 
discuss some current challenges related to numerical solutions and mathematical modeling with DDEs.



\end{document}
