\documentclass[12pt]{article}
\usepackage{lingmacros}
\usepackage{tree-dvips}
\begin{document}

Part I

Qualitative and Quantitative Features of Delay Differential Equations

Delay differential equations (DDEs) are also referred to as time-delay systems, systems with after-effect, memory, time-delay, hereditary 
systems, equations with deviating argument, or differential-difference equations. They belong to the class of functional differential 
equations that are infinite-dimensional, as opposed to ordinary differential equations (ODEs).

Recently, this class of differential equations has received considerable attention from researchers because the introduction of memory terms 
in a differential model significantly increases its complexity. Therefore, studying qualitative and quantitative behavior, numerical 
treatment of such models, parameter estimation, and sensitivity and stability analyses of delay integro-differential equations as well as 
stochastic delay differential equations (SDDEs) are essential. In this part (Chaps. 1–7), we will study the qualitative and quantitative 
features of DDEs, which have not been adequately investigated in the literature until now.

Chapter 1

Qualitative Features of Delay Differential Equations

1.1 Introduction

Ordinary and partial differential equations have long played an important role in bioscience, and they are considered to continue to serve as 
indispensable tools in future investigations as well. However, they frequently provide only a first approximation of the systems under 
consideration. More realistic models need to include some of the past states of these systems as well; that is, a real system needs to be 
modeled using differential equations with time-delays (or time-lags). Delay models formulated in mathematical biology include several types 
of functional differential equations, such as delay differential equations (DDEs), neutral delay differential equations (NDDEs), 
integro-differential equations, and retarded partial differential equations (RPDEs). Recently, stochastic delay differential equations 
(SDDEs) have attracted significant attention from researchers.

To create more realistic mathematical models for problems with time-lag or aftereffect, we need to consider using retarded functional 
differential equations (RFDEs) in place of ordinary differential equations (ODEs), such as

t, y(t), y(a(t, y(t))), (

t

y' (t) = f

 −∞

K(t, s, y(t), y(s))ds

)

,

t ≥ t0 ,

(1.1)

where a(t, y(t)) ≤ t and y(t) = ψ(t), t ≤ t 0 . Such retarded equations form a class of equations that is, in some sense, between ODEs and 
time-dependent partial differential equations (PDEs), and they generate infinite-dimensional dynamical systems. RFDEs (1.1), where the 
integral term is absent, are usually called delay differential equations (DDEs) and they assume forms such as

y' (t) = f (t, y(t), y(a(t, y(t)))) , a(t, y(t)) ≤ t.

(1.2)

Neutral delay differential equations (NDDEs) are defined by equations of the form

y' (t) = f ( t, y(t), y(a(t, y(t))), y' (β(t, y(t))) ) ,

(1.3)

where a(t, y(t)), β(t, y(t)) ≤ t. The introduction of the “lagging” or “retarded” argumentsaα(t, y(t)), β(t, y(t)) is to reflect an 
“after-effect”; e.g., the gestation period in population modeling.

Mathematical modeling of several real-life phenomena in bioscience requires “differential equations” that depend partially on the past 
history rather than only the current state. Such examples occur in population dynamics (taking into account the gestation and the maturation 
time), infectious diseases (accounting for the incubation periods), physiological and pharmaceutical kinetics (modeling, for example, 
hematopoiesis and respiration, where the delays are, respectively, due to cell maturation and blood transport between the lung and brain, 
etc.), chemical and enzyme kinetics (such as mixing reactants), biological immune response (in which the antibody production by the B-cell 
population depends on the antigenic stimulation at an earlier time), navigational control of ships and aircraft (with large and short lags, 
respectively), and more general control problems. An early use of DDEs was to describe technical devices such as control circuits. In that 
context, the delay is a measurable physical quantity; e.g., the time that the signal travels to the controlled object, the reaction time, and 
the time that the signal takes to return. Similarly, there are parallels in the reaction of the body to pain, for example. Refer to [1–11] 
for further examples of DDEs in biomathematics.

In many applications in the field of life sciences, a time-delay is introduced when there are certain hidden variables and processes that are 
not well understood but are known to cause a time-lag [12]. Thus, a delay may, in fact, represent a reaction chain or a transport process. We 
shall see later that the mathematical properties of DDEs justify such approximations. A well-known example is Cheyne-Stokes respiration (or 
periodic breathing), discovered in the nineteenth century, wherein some people show periodic oscillations of breathing frequency under 
constant conditions [13, 14]. This strange phenomenon is due to a delay caused by cardiac insufficiency in the physiological circuit 
controlling carbon dioxide levels in the blood.

Time-delays occur naturally in biological systems, e.g., in a chemostat (a laboratory device for controlling the supply of nutrients to a 
growing cell population

[15]). The use of ODEs to model a chemostat carries the implication that changes occur instantaneously. This is a potential deficiency of the 
ODE model. There are two sources of delays in the chemostat model: (i) delays due to the possibility that the organism stores the nutrients 
(so that the “free” nutrient concentration does not reflect the nutrients available for growth) and (ii) delays due to the cell cycle; see 
[16–18].

When delays are introduced in first-order non-linear differential equations, or in discrete difference equations, erratic solutions can 
appear (such a chaotic behavior is also observed in nature); see [19, 20] and Fig. 1.1.

Among the classical references for DDEs are the books by Bellman and Cooke

[21] and Elsgol’ts and Norkin [22]. These are rich sources for analytical techniques

and understanding the many interesting examples. Kolmanovskii et al. [23, 24] gave a rigorous treatment for a wide range of problems. The 
monographs of Hale [25] and Hale and Verduyn Lunel [26] are standard sources for understanding the theory of delay equations. Another 
important monograph is by Diekmann et al. [27]. Kuang

[28] and Banks [29] pay particular attention to problems in population dynamics, wherein the former looked at neutral equations. Gopalsamy 
[30] and Györi and Ladas

[31] addressed the question of oscillations in DDEs. Early books by Cushing [32], Driver [33], Halanay [34], MacDonald [18, 35], May [20], 
and Waltman [36] have been very stimulating for the development of the field.

Our concern in this chapter is with the qualitative features of DDEs. We show that DDEs have a richer mathematical framework for the analysis 
of biosystem dynamics compared with ODEs. First, we start from simple real-life problems and formulate them in terms of DDEs; see Sect. 1.2. 
We then briefly study the stability of delay models described by linear and non-linear DDEs, and the conditions that ensure stable behavior; 
see Sects. 1.3, 1.4 and 1.5.

1.2 Delay Models in Population Dynamics

In this section, we briefly discuss some simple mathematical models with time-delays of population dynamics. Naturally, the growth of a 
population subject to maturation delay is modeled by using either a discrete delay or a delay continuously distributed over the population. 
The use of a discrete delay might be seen as a rough approximation in modeling the delay distribution over a large population size. However, 
it is much more realistic to assume the delay being continuously distributed by a continuous distribution function, with a mean delay equal 
to the discrete delay.

1.2.1 Logistic Equation with Discrete Delay

Let y(t) be the population of a certain species that is independent of other species. The simple model of exponential growth is

y' (t) = λy(t) (λ > 0).

(1.4)

From the hypothesis that the growth rate will decrease with increasing population y(t) due to lack of resources (food and space), one arrives 
instead at the deterministic model of Verhulst (1845)

y(t) y' (t) = ry(t) 1 , ( K )

(1.5)

where λ in (1.4) is replaced by r

t → ∞.

If we now assume that the growth rate depends on the population of the preceding generation and take into account the hatching and maturation 
periods, then the above equation is replaced by a delay equation. Hutchinson (1948) [37] was one of the first to introduce a delay in a 
biological model. He modified the classical logistic equation (1.5) into the form

y(t − τ) '′ (t) = ry(t) 1 − . ( K )

(1.6)

Here,thederivativedependson y(t)andtheearlierstate y(t − τ),wherethelagτ > 0 represents the maturation time of individuals in the 
population. The non-negative parameters r and K are known as the intrinsic growth rate and the environmental carrying capacity, respectively.

Now, we illustrate how the presence of a delay in a differential equation can lead to a notable increase in the complexity of the observed 
behavior (stable steady states may be destabilized and consequently large amplitude oscillations can occur [38].) Consider a delayed logistic 
equation (1.6), which can be changed (by putting K y(t) = y(tτ), a  = bτ) into the form

dy(t) = a  y(t) [ 1 − y(t − 1) ] . dt

(1.7)

It is observed that the qualitative picture (Fig. 1.1) of the solution set of Eq. (1.7) is significantly dependent upon the delay parameter τ 
and upon the initial function. For “large” values of τ, the equation possesses undamped oscillatory solutions; whereas for small values of τ, 
the equation behaves like an ODE. For 0 < a  < π/2, x = 1 is a stable steady state; but for a  > π/2, chaotic behavior and periodic solution 
can

arise [39]. For a small a  − π/2, Morris [40] proved that the period is approximately p ∼ 4 + 1a(α  − π/2) [ π(3π − 2 ] . We note from 
Fig. 1.1a that the stable periodic solution of (1.7) rapidly acquires a spiky form as α  increases; see Fowler [38]. The numerical solution 
at α  = 3.5 consists of a series of well-separated pulses. This simple example illustrates many of the complexities that arise with delays 
and has the advantage that results may be easily and explicitly worked out.

1.2.2 Logistic Equation with Distributed Delay

Although Hutchinson’s approach leading to Eq. (1.6) is quite useful to explain the appearance of sustained oscillations in a single-species 
population without any predatory interaction of other species, the underlying argument is somewhat questionable. We may ask: How can it be 
that the present change in population size depends exactly on the population size of time τ units earlier? The question has led people to 
consider integro-differential equations [41]

t 1 y' (t) = ry(t) 1 − y(s)G(t − s)ds , t ≥ t 0 . (K ∫ t−τ )

(1.8)

Here, the derivative depends on y(t) and all the previous states after the initial moment t 0 . The delay is continuously distributed and the 
problem is said to have a fixed time-lag (or finite-memory) and a bounded retardation because the difference between t and t − τ isfifixed 
and bounded.

MacDonald [35] used the integro-differential equation

t y(t) y' (t) = ry(t) 1 − y(s)G(t − s)ds , {  ∫ 0 }

(1.9)

for parasite population growth that completes its life cycle within the same host and does not kill the host. (Immunological resistance by 
the host depends on exposure to the parasite population.) The delay here is continuously distributed and the problem is said to have an 
unbounded time-lag because the difference between 0 and t is unbounded. The initial time (t = 0) represents the start of the experiment or 
the time at which the naive host ingests the parasite. Here, it is possible to adopt the simple memory function G(t) = constant.

1.2.3 Delayed Lotka-Volterra System

Many mathematical studies using delay models to study ecology are built upon various generalizations of Volterra’s integro-differential 
system with infinite delays, which are motivated by the characteristic nature of predator-prey dynamics, such as

x (t) = b 1 ( ∫ −∞ y(s)k1 x(t) 1 − c 11 x(t) − c 12 (t − s)ds ) ,

(1.10)

t' y (t) = b 2 ( + ∫ −∞ x(s)k2 y(t) −1 c 21 (t − s)ds ) ,

where the variables x(t), y(t) represent the populations of the prey and the predator, and the parameters specifying the birth and 
interaction rates are non-negative 1 (see [32]).

In studying a similar interaction for predator-prey models, Wanggersky and Cunningham (1975) have used equations such as

x ′ (t) = ax(t) m−x(t) m bx(t)y(t), ( )

(1.11)

y ′ (t) = −cy(t) + dx(t − τ)y(t − τ).

More general delayed predator-prey models take the form

x ′ (t) = x(t)F(t, x t , y t ), y ′ (t) = y(t)G(t, x t , y t ),

(1.12)

where x t (θ) = x(t + θ), y t (θ) = y(t + θ) for θ ≤ 0, and F, G satisfy appropriate conditions (namely, ∂F/∂x t ≤ 0, ∂F/∂y t < 0; 
∂G/∂x t > 0, and ∂G/∂y t ≤ 0), and (1.12) has positive solutions.

A question of great importance is how does the qualitative behavior depends on the form and magnitude of the delays? In other words, are 
discrete and continuous delays equivalent from the perspective of the qualitative dynamical properties of the model? The paper by [12] 
examines certain aspects of this question.

In the next two sections, we discuss the stability of different types of DDEs.

1.3 Stability of DDEs

Time-delay is, in many cases, a source of instability. However, for some systems, the presence of delay can have a stabilizing effect. In the 
well-known example

y ′′ (t) + y(t) − y(t − τ) = 0,

(1.13)

the system is unstable for τ = 1, but it is asymptotically stable when τ = 1. The approximation y ′ (t) ≈ [ y(t) − y(t − τ) ] /τ explains 
the damping effect. The stability analysis and robust control of time-delay systems are, therefore, of theoretical and practical importance.

In the following subsections, we present a brief summary of some theories and analysis about the stability of linear and non-linear DDEs. We 
should first mention

the physical and mathematical interpretations of local and global stability. Local stability of an equilibrium point means that if you put 
the system somewhere near the point, then it will move itself to the equilibrium point in some time. However, global stability means that the 
system will come to the equilibrium point from any possible starting point (i.e., there is no “nearby” condition). Moreover, in local 
asymptotic stability, the solutions of the system must approach an equilibrium point under initial conditions close to the equilibrium point. 
Whereas in global asymptotic stability, the solutions must approach an equilibrium point under all initial conditions.

1.3.1 Stability of Linear Constant Coefficient DDEs

Consider a simple delay model of population growth given by the following linear DDE:

y ′ (t) = λy(t) + μy(t − τ), t ≥ t 0 , y(t 0 ) = ψ(t), t ≤ t 0 .

(1.14)

One of the fundamental methods for finding the solution of (1.14) is to build up the solution as a sum of simple exponential terms. Assuming 
the solution to be of the form y(t) = ce st (where c, and s are constants), it will be a solution of (1.14) if and only if s is a zero of the 
transcendental function

h(s) = s − λ − μe −sτ .

(1.15)

(The equation h(s) = 0 is called the characteristic equation of (1.14), and s r is the characteristic root if it is a zero of this equation.) 
Bellman and Cooke [21] observed that the roots s r of (1.15) are infinite in number and complex conjugate and that all lie in the left 
half-plane Re(s) < c, for some constant c.

Here, we summarize the necessary and sufficient conditions for the “asymptotical” stability of the linear DDEs (1.14). Driver [33], in the 
following theorem, provided the conditions for DDE (1.14) to be stable:

Theorem 1.1 A necessary and sufficient condition for all continuous solutions of (1.14) to approach zero as t → ∞ is that all the 
characteristic roots have negative real parts.

The following results impose conditions on λ and μ in (1.15) for the roots of h(s) = 0 to have negative real parts (Re(s) < 0):

• When λ and μ are complex. This case is also considered by Barwell [42] and he proved that: A sufficient condition that all the roots of 
(1.15) have negative real parts is

| μ | ≤ −Re(λ).

(1.16)

• When λ and μ are real, all roots of equation (1.15) have negative real parts if and only if (i) λ < 1, (ii) λ < −μ √ ζ 2 + λ 2 , where ζ 
is the root of ζ = λ tan(ζτ) such that 0 < ζτ < π (if λ = 0, take ζ = 1 2 π/τ); see Bellman and Cooke [21].

• When λ = 0 and μ is complex. This case has been considered by Barwell [42], and the result is: For μ = re iφ , a sufficient condition that 
all the roots of (1.15) have negative real parts is (i) Re(μ) < 0 ( 1 2 π < φ < 3 2 π), (ii) 0 < rτ < min( 3 2 π −

φ, φ − 1 2 π).

1.3.2 Asymptotical Stability Region for Linear DDEs

To find the asymptotical stability region [24] (which depends on the lag term τ), suppose, without any loss of generality, that τ = 1 in 
(1.14). We search for (λ, μ) values for which the first solution s crosses the imaginary axis (Re(s) = 0), i.e., s = iθ for θ real. If we 
insert this into (1.15), we obtain

λ = −μ for θ = 0 (s real),

λ = iθ − μe −iθ for θ = ̸ 0.

θ cos θ θ Byseparatingtherealandimaginaryparts,wegetλ = , μ = validfor sin θ sin θ

all real λ and μ. Thus, the stability region of y ′ (t) = λy(t) + μy(t − 1) is bounded by μ = −λ and the parametrized curve λ = θcot(θ), μ 
= −θ/sin(θ); see Fig. 1.2.

A smaller subset of the stability region, which has been classically considered in [42], is given by the set of pairs (λ, μ) such that the 
solution y(t) of (1.14) asymptotically vanishes independently of the lag τ (in the (λ, μ)-plane:

= { (λ, μ) ∈ R 2 | λ + | μ | < 0 } ).

We next extend this analysis to linear neutral DDEs.

1.3.3 Stability of Linear NDDEs

Consider a linear neutral delay differential equation of the form

y ′ (t) = λy(t) + μy(t − τ) + νy ′ (t − τ), t ≥ t 0 , y(t 0 ) = ψ(t), t ≤ t 0 .

(1.17)

We summarize the necessary and sufficient conditions for the stability of linear NDDEs (1.17) as follows:

Theorem 1.2 Every solution (of the form y(t) = ce st ) of (1.17) tends to zero as t → ∞ if all roots of the characteristic equation

s = λ + μe −τs + νse−τs 

(1.18)

have negative real parts and are bounded away from the imaginary axis.

Bellen et al. [43] gave a sufficient condition for the stability of the test equation (1.17) in the following theorem.

Theorem 1.3 A sufficient condition for all the roots of (1.18) to have negative real parts is

| λ¯ν − ¯μ | + | λν + μ | < −2Re(λ).

(λ, μ, and ν are complex parameters.)

Remark 1.1 If λ, μ, and ν are real, then the condition | λ¯ν − ¯μ | + | λν + μ | < −2Re(λ) is equivalent to the condition | μ | < −λ and | 
ν | < 1. If λ and μ are complex and ν = 0, then the hypothesis of Theorem 1.2 reduces to | μ | < −Re(λ), which gives a sufcient condition 
for the stability of the test equation (1.14).

1.3.4 Asymptotic Stability Region for Linear NDDEs

Suppose that τ = 1 in Eq. (1.17). We need to search for the stability regions in terms of parameters (λ, μ) for which the first solution s of 
(1.18) crosses the imaginary axis (Re(s) = 0), i.e., s = iθ for θ real. By separating the real and imaginary parts, we obtain

μ= θν cot(θ) for θ = 0 + ; sin(θ)

(1.19)

λ = −μ cos(θ) − θν sin(θ) for θ = ̸ 0.

(1.20)

The stability regions for the NDDE (1.17) in the space of parameters (λ, μ) for ν = −0.9, −0.5, 0.5, 0.9 are shown in Fig. 1.3. Equation 
(1.17) is always unstable for | ν | > 1; see [24].

1.4 Stability of Non-linear DDEs and Contractivity Conditions

Consider a more general, non-linear DDE with a fixed time-lag τ

y ′ (t) = f (t, y(t), y(t − τ)), t ≥ t 0 , y(t) = ψ(t), t ≤ t 0 ,

(1.21)

where y ∈ [ t 0 , ∞ ] → C n , f : [ t 0 , ∞) × C n × C n → C n andψ ∈ [ t 0 − τ, t 0 ] → C n .

We wish to examine the effect that a small change in the initial conditions has on a solution. Thus, we consider another system, defined by 
the same function f (t, y, x) of (1.21) but with another initial condition:

z ′ (t) = f (t, z(t), z(t − τ)), t ≥ t 0 , z(t) = φ(t), t ≤ t 0 .

(1.22)

In the sense of Lyapunov [24], the stability of the solution of (1.21) is defined by the following definition:

Definition 1.1 If there exists a norm on C n such that for every t ≥ t 0 , the solution of (1.21) is said to be

(1) stable (with respect to perturbing the initial function), if for each ε > 0 there exists δ = δ(ε, t 0 ) such that ∥ y(t) − z(t) ∥ ≤ ε 
when ∥ ψ(t) − φ(t) ∥ ≤ δ;

(2) asymptotically stable, if it is stable and ∥ y(t) − z(t) ∥ → 0 as t → ∞ ;

(3) uniformly asymptotically stable, if under condition (ii) the number δ = δ(ε) is independent of t 0 ;

(4) globally uniformly asymptotically stable, if δ can be an arbitrarily large, finite number;

(5) ξ-exponentially stable, if it is asymptotically stable and, given t 0 , there exists a finite constant K such that ∥ y(t) − z(t) ∥ ≤ 
Ke −ξ(t−t 0 ) ,

where y(t) and z(t) are solutions of (1.21) and (1.22), respectively, and ψ(t) and φ(t) are distinct and continuous functions.

Definition 1.2 The problem (1.21) is contractive (with respect to perturbing the initial function) if for every t ≥ t 0 :

∥ y(t) − z(t) ∥ ≤ max ∥ ψ(t) − φ(t) ∥ t ≤ t0 

holds.

Corollary 1.1 The zero solution of (1.21) is stable if there exists a norm on C n such that for every t ≥ t 0 :

∥ y(t) ∥ ≤ max ∥ ψ(t) ∥ . t ≤ t0 

The following theorem provides sufficient conditions for the contractivity of (1.21) (in the sense described above):

Theorem 1.4 (Contractivity Condition [44]) For a given inner product ⟨ ., . ⟩ in Cn  and the corresponding norm ∥ . ∥ , let σ(t) and γ (t) 
be continuous functions such that

σ(t) ≥

Re ⟨ ( f (t, y 1 , z) − f (t, y 2 , z), y 1 − y 2 ) ⟩ sup z,y 1 ,y 2 ∈C n ∥ y 1 − y 2 ∥2  y 1 ̸ =y 2

(1.23)

and

γ (t) ≥

∥ f (t, y, z 1 ) − f (t, y, z 2 ) ∥ sup y,z 1 ,z 2 ∈C n ∥ z 1 − z 2 ∥ z 1 =z ̸ 2

.

(1.24)

If

σ(t) + γ (t) ≤ 0, for every t ≥ t 0 ,

(1.25)

then it holds that

∥ y(t) − z(t) ∥ ≤ max ∥ ψ(x) − φ(x) ∥ , t ≥ t 0 . x ≤ to 

(1.26)

Corollary 1.2 Suppose that f (t, y(t), y(t − τ)) = λy(t) + μy(t − τ), as in Eq. (1.14). Then, σ(t) = Re(λ) and γ (t) = | μ | . In this 
case, if Re(λ) ≤ − | μ | , using theorem (1.4) we get | y(t) | ≤ max | ψ(t) | for every t ≥ t 0 .

t ≤ t0 

To prove Theorem 1.4, the following theorems are needed:

Theorem 1.5 Consider the initial value problems of the form

y ′ (t) = λ(t)y(t) + g(t), t ≥ t 0 , y(t 0 ) = y 0 ,

(1.27)

with y, λ, g : [ t 0 , + ∞) → C and Re(λ(t)) < 0 for every t ≥ t 0 . Then, the solution y(t) of the initial value problem (1.27) is such 
that:

| y(t) | ≤ max y 0 |; t 0 max x t | g(x)/(−Re(λ(x))) | . | { ≤≤ }

t0  Proof Define A(t) := ∫ t λ(x)dx; we note that Re(A(t)) < 0 for every t ≥ t 0 . The solution of (1.27) is

t y(t) = y 0 e A(t) + e A(t) e −A(x) g(x)dx. ∫ t0 

We have that

t t e −Re(A(x)) g(x)dx = −Re(A(x)) g(x)/(−Re(λ(x))) | | −Re(λ(x))e dx | ∫ ∫ [ ] | t 0 t 0

t ≤ max {| g(x)/(−Re(λ(x))) |} | −Re(λ(x))e −Re(A(x)) dx | , t 0 ≤ x ≤ t ∫ t 0

and

t −Re(λ(x))e −ReA(x)) dx = e −Re(A(t)) − 1. ∫ t0 

Therefore,

| e −Re(A(x)) g(x)dx | ≤ t 0 max x t { g(x)/(−Re(λ(x))) } | e −Re(A(t)) − 1 | . ∫ t 0 ≤ ≤

Hence:

| y(t) | ≤ e Re(A(t)) | y 0 | + (1 − e Re(A(t)) ) t 0 max x t | g(x)/(−Re(λ(x)) | , ≤ ≤

and so, for every t ≥ t 0 :

| y(t) | ≤ max y 0 |; t 0 max x t | g(x)/(−Re(λ(x))) | . | { ≤≤ }

Theorem 1.6 Consider, two initial value problems,

y ′ (t) = f (t, y(t), u(t)), t ≥ t 0 , y(t 0 ) = y 0 ,

(1.28)

and

z ′ (t) = f (t, z(t), v(t)), t ≥ t 0 , z(t 0 ) = z 0 ,

(1.29)

with f : [ t 0 , + ∞) × C n × C n → C n and y, z, u, v: [ t 0 , + ∞) → C n , and y 0 = ̸ z 0 . Assume there exists an inner product ⟨ ., . 
⟩ on C n such that (1.25) holds ( ∥ x ∥ = ⟨ x, x ⟩ for every x ∈ C n ). Then, for every t ≥ t 0 :

∥ y(t) − z(t) ∥ ≤ max y 0 − z 0 ∥; t 0 max x t { γ (x) ∥ u(x) − v(x) ∥ /(−σ(x))) } . ∥ { ≤≤ }

Proof We have

1 d y(t) z(t) = Re ⟨ y ′ (t) − z ′ (t), y(t) − z(t) ⟩ ∥ ∥2  2 dt

= Re ⟨ f (t, y(t), u(t)) − f (t, z(t), v(t)), y(t) − z(t) ⟩

= Re ⟨ f (t, y(t), u(t)) − f (t, y(t), v(t)), y(t) − z(t) ⟩ +

Re ⟨ f (t, y(t), v(t)) − f (t, z(t), v(t)), y(t) − z(t) ⟩ .

It follows from the definitions of σ(t) and γ (t) and from Schwartz inequality that

1 d y(t) z(t) ≤ ∥ f (t, y(t), u(t)) − f (t, y(t), v(t)) ∥∥ y(t) − z(t) ∥ + σ(t) ∥ y(t) − z(t) ∥2  ∥ ∥2  2 dt

≤ γ (t) ∥ u(t) − v(t) ∥∥ y(t) − z(t) ∥ + σ(t) ∥ y(t) − z(t) ∥ 2 .

Define

Y(t) := ∥ y(t) − z(t) ∥ .

Note that Y(t) > 0 for every t > t 0 because we assume that the function f is such that (1.28) has a unique solution y(t) for every initial 
condition y(t 0 ) = y 0 .

Then,

1 d ′ d = ∥ y(t) − z(t) ∥ (t), ∥ y(t) − z(t) ∥ 2 ∥ y(t) − z(t) ∥ = Y(t)Y 2 dt dt

so we have

Y(t)Y ′ (t) ≤ σ(t)Y 2 (t) + γ (t) ∥ u(t) − v(t) ∥ Y(t),

and hence

Y ′ (t) ≤ σ(t)Y(t) + γ (t) ∥ u(t) − v(t) ∥ .

Define g(t) := γ (t) ∥ u(t) − v(t) ∥; Therefore,

Y ′ (t) ≤ σ(t)Y(t) + g(t),

and, by Theorem 1.5, for t ≥ t 0 :

Y(t) ≤ max Y 0 ; max g(x)/(−σ(x)) , { t 0 ≤ x ≤ t }

i.e.,

∥ y(t) − z(t) ∥ ≤ max y 0 − z 0 ∥; t 0 max x t { γ (x) ∥ u(x) − v(x) ∥ /(−σ(x))) } . ∥ { ≤≤ }

Proof Theorem 1.4. From Theorem 1.6 we know that, for every t ≥ t 0 , the solutions y(t) and z(t) of (1.21) and (1.22), respectively, are 
such that

∥ y(t) − z(t) ∥ ≤ max { ∥ ψ(t 0 ) − φ(t 0 ) ∥; max t 0 ≤ x ≤ t γ (x) ∥ y(x − τ) − z(x − τ) ∥ /(−σ(x)) } .

Assume that γ (t) ≤ −σ(t) and τ > 0 for every t ≥ t 0 ; therefore:

∥ y(t) − z(t) ∥ ≤ max ψ(t 0 ) − φ(t 0 ) ∥; t 0 max x t ∥ y(x − τ) − z(x − τ) ∥ , ∥ { ≤≤ }

i.e.,

∥ y(t) − z(t) ∥ ≤ max {∥ ψ(t) − φ(t) ∥} . t ≤ t0 

Therefore, the DDE (1.21) is stable if conditions (1.23)–(1.25) are satisfied.

Next, we will study global stability using Lyapunov functionals.

1.5 Stability of DDEs in Lyapunov Method

Lyapunov functions are an essential tool in the stability analysis of dynamical 
systems,bothintheoryandapplications.Asinsystemswithoutdelay,anefficientmethod for stability analysis of DDEs is the Lyapunov method. For 
DDEs, there exist two main Lyapunov methods: the Krasovskii method of Lyapunov functionals [45] and the Razumikhin method of Lyapunov 
functions [46, 47]. The two Lyapunov methods for linear DDEs result in linear matrix inequalities (LMIs) conditions. The LMI approach to 
analysis and design of DDEs provides constructive finite-dimensional conditions, despite significant model uncertainties [48].

Consider a simple DDE of the form

y ′ (t) = f (t, y(t − τ)), t ≥ t 0 ,

(1.30)

where f : R × C [ −τ, 0 ] → R n is continuous in both arguments and is locally Lipschitz continuous in the second argument. We assume that 
f (t, 0) = 0, which guarantees that (1.30) possesses a trivial solution y(t) = 0. The system is uniformly asymptotically stable if its 
trivial solution is uniformly asymptotically stable.

The core concept of Lyapunov stability theory is to construct a functional V (y(t)) (total energy stored in a system) to be defined and its 
derivative along the trajectories of the system.

Definition 1.3 Let V : R n → R be a Lyapunov function if

(i) V (y(t)) ≥ 0 with equality if and only if y = 0, and

(ii) dt d V (y(t)) ≤ 0.

Theorem 1.7 (Lyapunov’s Second Theorem on R) If there exists a Lyapunov function V , then y = 0 is Lyapunov stable. Furthermore, if V (y(t)) 
< 0, then equilibrium y = 0 is asymptotically stable.

Given a DDE of the form:

y ′ (t) = f (y(t), y(t − τ)), f (0, 0) = 0,

where f (., .) is locally Lipschitz in its arguments. Let us assume that V (t) = y 2 (t), which is a typical Lyapunov function for n = 1. 
Then, we have along the system:

V ′ (t) = 2y(t)y ′ (t) = 2y(t) f (y(t), y(t − τ)).

For the feasibility of inequality V ′ (t) ≤ 0, we need to ensure that y(t) f (y(t), y(t τ)) ≤ 0 for all sufficiently small | y(t) | and | 
y(t − τ) | . This essentially restricts the class of equations considered. For example, y ′ (t) = −y(t)y 2 (t − τ) is stable based on the 
above arguments.

1.5.1 Lyapunov-Krasovskii Sense

Let V : R × C [ −τ, 0 ] → R be a continuous functional, and let y s (t, φ) be the solution of (1.30) at time s ≥ t with the initial 
condition y t = φ(t). We define the right ˙ upper derivative V (t, φ) along (1.30) as follows:

1 V ˙ (t, φ) = lim sup (t, φ) − V (t, φ) ] . V (t t, yt + t  + [ t→0 + t

˙ Intuitively, a non-positive (t, φ) indicates that yt  the system under consideration is stable.

does not grow with t, meaning that

Theorem 1.8 (Lyapunov-Krasovskii Theorem, Gu et al. [49]) Suppose that f : R × C [ −τ, 0 ] → R n maps R× (bounded sets) in C [ −τ, 0 ] 
into bounded sets of R n and that u ; v ; w : R + → R + are continuous nondecreasing functions, u(s) and v(s) are positive for s > 0, and 
u(0) = v(0) = 0. The trivial solution of (1.30) is uniformly stable if there exists a continuous functional V : R × C [ −τ, 0 ] → R + , 
which is positive-definite, i.e.,

u( | φ(0) | ) ≤ V (t, φ) ≤ v( | φ(0) | ),

(1.31)

and such that its derivative along (1.30) is non-positive in the sense that

V ˙ (t, φ) ≤ −w( ∥ φ ∥ C ).

(1.32)

If w(s) > 0 for s > 0, then the trivial solution is uniformly asymptotically stable. If in addition lim u(s) = ∞, then it is globally 
uniformly asymptotically stable.

s→∞

1.5.2 Lyapunov-Razumikhin Sense

In Razumikhin approach, the derivative V along the solution y(t) of (1.30) of a differentiable function V : R × R n → R + is defined as 
follows:

d ∂V (t, y(t) ∂V (t, y(t)) V ˙ (t, y(t)) = + f (t, y t ). V (t, y(t)) = dt ∂t ∂y

(1.33)

Theorem 1.9 (Lyapunov-Razumikhin Theorem, Gu et al. [49]) Suppose that f : R × C [ −τ, 0 ] → R n maps R× (bounded sets) in C [ −τ, 0 ] 
into bounded sets of R n and that u ; v ; w : R + → R + are continuous nondecreasing functions, u(s) and v(s) are positive for s > 0, and 
u(0) = v(0) = 0, v is strictly increasing. The trivial solution of (1.30) is uniformly stable if there exists a continuous functional V : R × 
C [ −τ, 0 ] → R + , which is positive-definite, i.e.,

u( | y | ) ≤ V (t, y) ≤ v( | y | ),

(1.34)

and the derivative along (1.30) satisfies

V ˙ (t, y(t)) ≤ −w( | y(t) | ), i f V (t + θ, y(t + θ) < V (t, y(t)), f or θ ∈ [ 0, τ ] .

(1.35)

If, in addition, w(s) > 0 for s > 0, and there exists a continuous nondecreasing function ρ(s) > 0, for s > 0, such that condition (1.35) is 
strengthened to

V ˙ (t, y(t)) ≤ −w( | y(t) | ), i f V (t + θ, y(t + θ) < ρ(V (t, y(t))), f or θ ∈ [ 0, τ ] , (1.36)

then the trivial solution is uniformly asymptotically stable. If in addition lim u(s) = s→∞ ∞, then it is globally uniformly 
asymptotically stable.

1.5.3 Stability of Linear Systems with Discrete Delays

Given the linearized system

y˙ (t) = Ay(t) + By(t − τ(t)), y(t) = φ(t), t ∈ [ −τ M , 0 ] ,

(1.37)

τ(t) ∈ [ 0, τ ] is abounded. A simple Lyapunov-Krasovskii functional for the above system has the form

t V (t, y(t)) = y T (t)Py(t) + y(s)Qy(s)ds, ∫ t−τ(t)

where P > 0 and Q > 0 are n × n matrices. Clearly V satisfies the positivity condition V (t, y(t)) ≥ β | y(t) 2 , for β > 0. Then, 
differentiating V along the system, we have

V ˙ (t, y(t)) = 2y T (t)P ˙x(t) + y T (t)Qy(t) − (1 − ˙τ)y T (t − τ)Qy(t − τ).

If we further substitute y˙ (t), the right-hand side of the DDEs system, with τ˙ ≤ d ≤ 1, we arrive at

y(t) V yT  ˙ (t, (y(t)) ≤ (t)y T (t − τ)W 2 , for ε > 0, if ≤ −ε | y(t) | [ y(t − τ) ]

A T P + PA + Q PB < 0. [ B T A −(1 − d)Q ]

(1.38)

The linear matrix inequality (LMI) (1.38) does not depend on τ and it is, therefore, delay-independent (but delay-derivative dependent). The 
feasibility of LMI (1.38) is a sufficient condition for the delay-independent asymptotic stability of systems with slowly varying delays; see 
[49].

However, delay-independent conditions cannot be applied for the stabilization of unstable plants through a feedback with delay. For such 
systems, delay-dependent conditions are then needed. Now, we derive stability conditions by applying Razumikhin’s approach and using the 
Lyapunov function:

V (t, y(t)) = y T (t)Py(t)

with P > 0 that satisfies the positivity condition (1.34). Consider the derivative of V 
along(1.37).WewillapplytheLyapunov-Razumikhintheoremwithρ(s) = ¯ρ.s > 1, where the constant ¯ρ > 1. Whenever Razumikhin’s condition:

¯ρy T (t)Py(t) − y T (t − τ(t))Py(t − τ(t)) > 0

holds for ¯ρ + 1, with  > 0. We then conclude that, for any q > 0, there exists α > 0 such that

V ˙ (t, y(t)) =2y T (t)P [ Ay(t) + A 1 y(t − τ(T )) ]

(1.39)

≤ 2y T (t)P [ Ay(t) + A 1 y(t − τ) ] +

q [ ¯ρy T (t)Py(t) − y T (t − τ(t))Py(t − τ(t)) ] ≤ −α | y(t) | 2

if

A T P + PA + qP PB < 0. [ B T P −q P ]

(1.40)

The MLI (1.40) does not depend on τ. Therefore, the feasibility of (1.40) is sufficient for delay-independent uniform asymptotic stability 
for systems with fast-varying delays (without any constraints on the delay-derivatives); see [49].

1.6 Concluding Remarks

In this chapter, we have provided a general introduction on DDEs and examined the stability of delay models described by linear and 
non-linear DDEs along with conditions that ensure local and global asymptotic stable behavior. Next, we will study approximation solutions 
and numerical schemes of DDEs. We will also discuss how the Runge-Kutta methods, which are so popular for ODEs, can be extended to DDEs.

References

1. Balasubramaniam, P., Prakash, M., Rihan, F.A., Lakshmanan, S.: Hopf bifurcation and stability of periodic solutions for delay differential 
model of HIV infection of CD4 + T-cells. Abstr. Appl. Anal. 982978, 1–18 (2014)

2. Bocharov, G.A., Rihan, F.A.: Numerical modelling in biosciences using delay differential equations. J. Comput. Appl. Math. 125, 183–199 
(2000)

3. Metz,J..A..J.,Diekmann,O.: The DynamicsofPhysiologicallyStructuredPopulations.Lecture Notes in Biomathematics, vol. 68. Springer, NY 
(1986)

4. Rihan, F.A.: Numerical treatment of delay differential equations in bioscience. PhD. Thesis, University of Manchester, UK (2000)

5. Rihan, F.A., Abdelrahman, D., Al-Maskari, F., Ibrahim, F.: A delay differential model for tumour-immune response and control with 
chemo-immunotherapy. Comput. Math. Methods Med. 2014, 15 (2014)

6. Rihan, F.A., Abdelrahman, D.H., Lakshmanan, S.: A time delay model of tumour- immune system interactions: Global dynamics, parameter 
estimation, sensitivity analysis. Appl. Math. Comput. 232, 606–623 (2014)

7. Rihan, F.A., Azamov, A.A., AlSakaji, H.J.: An inverse problem for delay differential equations:

parameter estimation, nonlinearity, sensitivity. Appl. Math. Inform. Sci. 12(1), 63–74 (2018)

8. Rihan, F.A., Lakshmanan, S., Maurer, H.: Optimal control of tumour-immune model with time-delay and immuno-chemotherapy. Appl. Math. 
Comput. 353(7), 147–165 (2019)

9. Rihan, F.A., Rihan, B.F.: Numerical modelling of biological systems with memory using delay differential equations. Appl. Math. Inf. Sci. 
9(3), 1615–1658 (2015)

10. Rihan, F.A., Kuang, Y., Bocharov, G.: Delay differential equations: theory, applications and new trends, vol. 13. Editorial: Discrete and 
Continuous Dynamical Systems - Series S (2018)

11. Rihan, F.A., Tunc, C., Saker, S.H., Lakshmanan, S., Rakkiyappan, R.: Applications of delay differential equations in biological systems, 
vol. 2018. Editorial: Complexity (2018)

12. Cooke, K., Grossman, Z.: Discrete delays, distributed delays and stability switches. J. Math.

Anal. Appl. 86, 592–624 (1982)

13. Mackey, M.C., Glass, L.: Oscillations and chaos in physiological control systems. Science 197, 287–289 (1977)

14. Mackey, M.C., Milton, J.C.: Feedback, delays, and the origin of blood cell dynamics. Comm.

Theor. Biol. 1, 299–327 (1990)

15. Smith, H.L., Waltman, P.: The Theory of the Chemostat. Cambridge University Press, Cambridge (1994)

16. Bertta, E., Bischi, G., Solimano, F.: Stability in chemostat equations with delayed nutrient recycling. J. Math. Biol. 28, 99–111 (1990)

17. Caperon, R.P.: Time lag in population growth response of isochrysis galbana to variable nitrate environment. Ecology 50, 188–192 (1969)

18. Caperon, R.P.: Biological Delay System: Linear Stability Theory. Cambridge University Press, Cambridge (1989)

19. Lorenz, E.N.: Deterministic non-periodic flow. J. Atmos. Sci. 20, 130–141

20. May, R.: Stability and Complexity in Model Ecosystems. Princeton University Press, Princeton, New Jersey (1974)

21. Bellman, R., Cooke, K.L.: Differential-Difference Equations. Academic Press, New York (1963)

22. Elsgolt’s, L.E., Norkin, S.B.: Introduction to the theory and application of differential equations with deviating arguments

23. Kolmanovskii, V.B., Myshkis, A.: Applied Theory of Functional Differential Equations (1992)

24. Kolmanovskii, V.B., Nosov, V.R.: Stability of Functional Differential Equations. Academic Press, NY (1986)

25. Hale, J.: Theory of Functional Differential Equations. Springer, New York (1997)

26. Hale, J.K., Verduyn Lunel, S.M.: Introduction to Functional Differential Equations. Springer, NY (1993)

27. Diekmann, O., van Gils, S., Verduyn Lunel, S., Walter, H.-O.: Delay Equation, Functional-, Complex-, and Nonlinear Analysis. Springer, 
Berlin (1995)

28. Kuang, Y.: Delay Differential Equations with Applications in Population Dynamics. Academic Press (1993)

29. Banks, R.B.: Growth and Diffusion Phenomena. Mathematical Frameworks and Applications.

Texts in Applied Mathematics, vol. 14. Springer, Berlin (1994)

30. Gopalsamy, K.: Stability and Oscillations in Delay Differential Equations of Population Dynamics. Kluwer, Dordrecht (1992)

31. Györi, I., Ladas, G.: Oscillation Theory of Delay Equations with Applications. Oxford Mathematical Monographs. Clarendon Press, Oxford

32. Cushing, J.M.: Integro-Differential Equations and Delay Models in Population Dynamics.

Lecture Notes in Biomathematics. Springer, Berlin (1977)

33. Driver, R.D.: Ordinary and Delay Differential Equations. Applied Mathematics Series 20.

Springer (1977)

34. Halanay, A.: Differential Equations, Stability, Oscillations, Time Lags. Academic Press, New York (1966)

35. MacDonald, N.: Time-Lags in Biological Models. Lecture Notes in Biomathematics, vol. 27.

Springer, Berlin (1978)

36. Waltman, P.: Deterministic Threshold Models in the Theory of Epidemics. Lecture Notes in Biomathematics, vol. 1. Springer, Berlin (1974)

37. Hutchinson, G.E.: Circular casual systems in ecology. Anal. New York Acad. Sci. 50, 221–246 (1948)

38. Fowler, A.C.: An asymptotic analysis of the delayed logistic equation when the delay is large.

IMA J. Appl. Math. 28(1), 41–47

39. Jones, G.S.: The existence of periodic solutions of f ′ (x) = −αf (x − −1) { 1 + f (x) } . J. Math.

Anal. Appl. 5, 435–450 (1962)

40. Morris, H.C.: A perturbative approach to periodic solutions of delay-differential equations. J.

Inst. Math. Applics. 18, 15–24 (1976)

41. Morris, H..C.: Variations and fluctuations in the numbers of co-existing animal species. In:

Scudo, F.M., Ziegler, J.R. (eds.) The Golden Age of Theoretical Ecology: 1923-1940. Lecture Notes in Biomathematics, vol. 22. Springer, 
Berlin (1979)

42. Barwell, V.K.: Special stability problems for functional equations, pp. 130–135 (1975)

43. Bellen, A., Jaciewicz, Z., Zennaro, M.: Stability analysis of one-step methods for neutral delaydifferential equations. Numer. Math. 52, 
605–619 (1988)

44. Bellen, A., Jaciewicz, Z., Zennaro, M.: Stability of numerical methods for delay differential equations. J. Comput. Appl. Math. 25, 15–26 
(1989)

45. Krasovskii, N.: Stability of Motion. Stanford University Press (1959)

46. Lyapunov, A.M.: The general problem of the stability of motion. Int. J. Control 55(3), 531–534 (1992)

47. Razumikhin, B.: On the stability of systems with a delay. Prikl. Math. Mech. (in Russian) 20

48. Boyd, S., El Ghaoui, L., Feron, E., Balakrishnan, V.: Linear matrix inequality in systems and control theory. SIAM, Studies in Applied 
Mathematics, vol. 15. Philadelphia (1994)

49. Gu, K., Kharitonov, V., Chen, J.: Stability of Time-delay Systems. Birkhauser Boston, USA (2003)



\end{document}
